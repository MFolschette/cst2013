\chapter{Autres activités}

\newcommand{\enseignement}[5]{\entree{}%$\bullet$}
{#1}{#2}{%
\ifnotempty{#3}{#3\ifnotempty{#4}{\\}}%
\ifnotempty{#4}{\textbf{Responsable :} #4}%
\begin{itemize}\renewcommand{\labelitemi}{\qquad$\circ$}#5\end{itemize}}}

\newcommand{\formation}[3]{\entree{$\bullet$}{#1}{(#2)}{#3}}

\section{Enseignements dispensés durant l'année scolaire 2012/2013}

J'ai effectué une activité complémentaire d'enseignement (monitorat) à l'École Centrale de Nantes auprès d'élèves de 1\textsuperscript{ère}, 2\textsuperscript{e} et 3\textsuperscript{e} année (Bac +3 à Bac +5). Les heures de TP sont décomptées comme équivalentes aux heures de TD.

\bigskip

\enseignement
{Méthodes logicielles (MELOG)}{2\textsuperscript{e} année (semestre~7)}
{Programmation orientée objet, structures de données et langage Java}
{Guillaume MOREAU}
{\item 1 groupe de TD, TP et TA, soit 28 heures}

\enseignement
{Algorithmique et programmation (ALGPR)}{1\textsuperscript{ère} année (semestre~6)}
{Introduction à l'algorithmique et applications au langage C}
{Vincent TOURRE}
{\item 1 groupe de TD, TP et TA, soit 30 heures}

\enseignement
{Projet d'application (PAPPL)}{3\textsuperscript{e} année, option informatique (semestre~8)}
{Projet informatique appliqué}
{Didier LIME}
{\item 1 projet de 3 étudiants en co-encadrement, soit environ 5 heures}

\enseignement
{Projet de groupe (PGROU)}{3\textsuperscript{e} année, option informatique (semestre~8)}
{Projet informatique en groupe}
{Guillaume MOREAU}
{\item 1 projet de 4 étudiants en co-encadrement, soit environ 4 heures}

\enseignement
{Projet de recherche et développement (R\&D)}{3\textsuperscript{e} année, option R\&D (semestre~8)}
{Projet à caractère scientifique}
{Ina TARALOVA}
{\item 1 projet d'1 étudiant en co-encadrement, soit environ 2 heures}

Ma charge d'enseignement pour cette année scolaire s'élève donc à environ 69~heures, pour 64~heures requises par mon contrat.
L'année dernière, j'avais effectué environ 68~heures d'enseignement.



\section{Projets encadrés durant l'année scolaire 2012/2013}

J'ai eu l'occasion d'encadrer trois projets d'étudiants durant cette année scolaire. Tous les étudiants concernés étaient en 3\textsuperscript{e} année à l'École Centrale de Nantes.

Deux groupes d'étudiants (PGROU et PAPPL ci-avant) et se sont concentré sur le projet d'interface graphique en développement gPH permettant de manipuler visuellement des modèles en Process Hitting et d'y appliquer des outils de Pint.
Ces deux projets ont permis d'aboutir à une version stable de l'interface permettant d'effectuer des tâches de base :
\begin{itemize}
  \item proposer une représentation graphique dynamique d'un modèle de Process Hitting,
  \item associer divers attributs graphiques (couleur, position) aux sortes et les répartir en groupes,
  \item éditer la version texte du modèle et permettre une mise à jour en direct de la version graphique,
  \item définir et appeler des branchements vers des outils extérieurs (de la bibliothèque Pint notamment).
\end{itemize}
L'objectif de cette application est de permettre l'utilisation du Process Hitting par des non-informaticiens ou d'offrir une démonstration visuelle du formalisme et de ses capacités.

Le dernier projet (R\&D ci-avant), mené par un seul étudiant, a porté sur la traduction d'un modèle de Thomas avec multiplexes vers un modèle de Thomas sans multiplexe, en vue de la traduction de celui-ci en réseau de Petri.
Ce projet a permis de soulever des problématiques intéressantes liées à ce problème précis, et la traduction a été formalisée et démontrée par l'étudiant.



\section{Modules d'apprentissage de l'école doctorale}

Cette deuxième année de doctorat a été l'occasion de compléter les formations requises par l'école doctorale,
et de rattraper le retard engendré l'an passé par le stage doctoral de trois mois au Japon.
L'école doctorale STIM requiert trois formations à caractère scientifique et trois formations à caractère professionnel,
et j'ai profité de ma participation aux différentes écoles thématiques pour valider une partie des formations scientifiques.

\subsection{Modules scientifiques}

\formation
{MOVEP'12}{en attente de validation}
{L'école de jeunes chercheurs MOVEP a pour but de réunir des chercheurs, des doctorants et des industriels travaillant dans les domaines du contrôle et de la vérification formelle de système parallèles. Elle se déroule sur une semaine (du 3 au 7 décembre) et se décline en tutoriels, présentations techniques et présentations d'étudiants.

Le thème de cette école de jeunes chercheurs s'inscrit bien dans mon champ de recherche, qui est l'application de méthodes formelle à des modèles biologiques algébriques. J'ai de plus eu la possibilité de présenter mon sujet et mon travail passé dans le cadre des présentations d'étudiants.

Plus d'informations à : \url{http://movep.lif.univ-mrs.fr/}}

\formation
{ASSB'13}{en attente de validation}
{L'école thématique ASSB a pour but de réunir des chercheurs dans le domaine de la biologie des systèmes et de la biologie synthétique. Elle se déroule sur une semaine (du 25 au 29 mars) et s'articule autour de conférences et d'ateliers.

Cette école thématique a été l'occasion d'entendre et de rencontrer des chercheurs du milieu dont j'avais parfois déjà lu des articles (bio-informatique et biologie des systèmes). Les exposés brassaient des thèmes assez variés, permettant notamment une ouverture vers la biologie que ma formation initiale ne m'avait pas offerte. Cela m'a aussi permis de prendre quelques contacts dans le milieu. Enfin, j'ai eu la possibilité de présenter mon sujet et mon travail passé dans le cadre des présentations d'étudiants.

Plus d'informations à : \url{http://epigenomique.free.fr/en/index.php}}

\formation
{JDOC'13}{validé}
{La Journée des Doctorants (JDOC) est une manifestation destinée à promouvoir les rencontres entre les doctorants de l'école doctorale et à échanger sur leurs travaux de thèse en cours de préparation avec l'ensemble des autres doctorants, directeurs de recherche et co-encadrants de l'école doctorale STIM. Elle concerne et implique la participation des doctorants de STIM inscrits en 2\textsuperscript{e} année ; c'est une journée d'échange et d'apprentissage de la communication vers un public de scientifiques non spécialisés.

Objectifs :
\begin{itemize}
  \item rapprocher les doctorants des différentes spécialités,
  \item faire connaitre et valoriser les travaux de chacun,
  \item favoriser les échanges.
\end{itemize}

Plus d'informations à : \url{https://sites.google.com/site/jdoc2013ang/}}



\subsection{Modules professionnels}

\formation
{Communication orale : présenter son projet de thèse en temps limité}{validé}
{Les objectifs de ce module sont les suivants :
\begin{itemize}
  \item Résumer en vingt minutes son projet de thèse au moyen de « Power Point », devant un auditoire pluridisciplinaire ;
  \item Préparation de l'exposé : apprendre à sélectionner ce qui est essentiel et important ; structurer son exposé pour capter et garder l'attention de son auditoire ;
  \item Apprendre à se mettre en scène et mise en situation ;
  \item Vidéo-scopie, analyses critiques, autoévaluation.
\end{itemize}

Ce module est l'occasion d'apprendre à s'adresser à un public divers et éventuellement non spécialisé (soutenance de thèse, réunion de laboratoire, entretien d'embauche, etc.).
Il m'a aussi permis de m'entraîner à présenter mon sujet devant un public provenant de disciplines diverses (droit, lettres, sciences sociales, biologie, STAPS et informatique).}

\formation
{Découverte du journalisme scientifique avec initiation pratique au journalisme radio}{en attente de validation}
{L'objectif principal de ce module est d'initier les jeunes chercheurs de toutes les disciplines au journalisme scientifique et à la vulgarisation des sciences par la pratique, \textit{via} la réalisation d'une émission radio

Outre la découverte du métier du journaliste scientifique et la sensibilisation à l'exercice de la vulgarisation, cette initiation au journalisme scientifique s'appuie sur toutes les différentes étapes de la réalisation d'une émission scientifique, commençant par la recherche, passant par l'écriture et l'enregistrement de voix à la diffusion d'une émission radio. Ce module de formation invite chaque participant à prendre du recul par rapport à ses travaux de recherche et par rapport à sa discipline lors de l'écriture d'articles, de réalisation d'interviews et de reportages sur un sujet choisi (thématique de recherche traitée dans un laboratoire de la région).

L'émission de radio réalisée lors de ce module, intitulée \og Comprendre le monde quantique \fg, a été diffusée dans \textit{Le Labo des Savoirs} de la radio nantaise Prun'.
Le podcast de l'émission est gratuitement disponible à l'écoute sur le site Internet de la radio,
à l'adresse : \url{http://www.prun.net/emissions/le-labo-des-savoirs-07052013}}

\formation
{Anglais pour la recherche}{en cours}
{L'objectif de ce module est de tendre vers l'autonomie de l'anglais pour les sciences et acquérir des connaissances et des compétences s'appliquant au domaine professionnel :
\begin{enumerate}
  \item Prise de parole en public :
  \begin{itemize}
    \item préparation à la présentation de travaux en congrès,
    \item présentation d'une communication de 15' sur son sujet de thèse ;
  \end{itemize}
  \item Révision des notions grammaticales et lexicales de base de l'anglais scientifique avec l'ouvrage \textit{Minimum Competence in Scientific English}.
\end{enumerate}}



\section{Activités bénévoles}
\begin{itemize}
\renewcommand{\labelitemi}{\qquad$\circ$}
  \item Membre actif de l'\emph{Association des Étudiants en Doctorat} sur le campus de l'École Centrale de Nantes (AED).
  \item Participation à la réalisation d'un cours d'initiation en trois séances à \LaTeX{} auprès des élèves en doctorat et master sur le campus de l'École Centrale de Nantes.
\end{itemize}
