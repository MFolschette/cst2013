\chapter{Activités scientifiques}

\section{Publications}\label{sec:publications}

\subsection{Publications 2012/2013}
\todo{CS2Bio}

\todo{JDOC ?}

\subsection{Publications précédentes}\label{ssec:publications}
\todo{Alléger}

\paragraph{Article accepté en conférence}
\begin{itemize}
\item[] \textbf{Maxime Folschette}, Loïc Paulevé, Katsumi Inoue, Morgan Magnin, Olivier Roux.
Concretizing the Process Hitting into Biological Regulatory Networks,
in : CMSB'12\!\!: Proceedings of the 10th International Conference on Computational Methods in Systems Biology,
London, UK, ACM, octobre 2012.
\end{itemize}
Une version préliminaire de cette publication est disponible en annexe de ce document.

\paragraph{Article soumis en workshop}
\begin{itemize}
\item[] \textbf{Maxime Folschette}, Loïc Paulevé, Katsumi Inoue, Morgan Magnin, Olivier Roux.
Abducting Biological Regulatory Networks from Process Hitting models,
in : LDSSB'12\!\!: ECML/PKDD 2012 Workshop on Learning and Discovery in Symbolic Systems Biology,
University of Bristol, UK, septembre 2012
\end{itemize}

\paragraph{Précédente publication en journal} dans le domaine de la fusion nucléaire
\begin{itemize}
\item[] Andrea Murari, Didier Mazon, Michela Gelfusa, \textbf{Maxime Folschette}, Thibaut Quilichini et collaborateurs EFDA-JET.
Residual analysis of the equilibrium reconstruction quality on JET, \textit{Nuclear Fusion},
volume 51, numéro 5, avril 2011, DOI 10.1088/0029-5515/51/5/053012.
\end{itemize}



\section{Exposés invités}

\subsection{Exposés avec proceedings}

\todo{MOVES}

\todo{ASSB}

\todo{Klamt ?}


\subsection{Exposés précédents}
\todo{Alléger}

\paragraph{Réunion du groupe de travail ANR BioTempo/G2 : « Des dynamiques discrètes à des dynamiques continues (modèles hybrides) »}
juin 2012, Nantes, France
\begin{itemize}
\item[] Concretizing Process Hitting models into Biological Regulatory Networks with Thomas' formalism using ASP
\end{itemize}

\paragraph{Séminaire informel AED}
juin 2012, Nantes, France
\begin{itemize}
\item[] Modeling and Analysis of Large Biological Regulatory Networks thanks to the “Process Hitting” Framework
\end{itemize}

\paragraph{Fourth CSPSAT \& ASP Seminar}
mai 2012, Kobe, Japon
\begin{itemize}
\item[] Concretizing Process Hitting models into Biological Regulatory Networks with Thomas' formalism using ASP
\end{itemize}

\paragraph{KUBIC-NII Joint Seminar on Bioinformatics 2012}
avril 2012, Kyoto, Japon
\begin{itemize}
\item[] Translating Process Hitting models to Thomas' modeling with ASP
\end{itemize}

\paragraph{The 8th Meeting on Inference-based Hypothesis-finding and its Application to Systems Biology}
mars 2012, Kanazawa, Japon
\begin{itemize}
\item[] Modeling and Analysis of Large Biological Regulatory Networks thanks to the Process Hitting Framework
\end{itemize}



\section{Collaboration}

\paragraph{Katsumi Inoue} National Institute of Informatics, Tokyo, Japon

Séjour AtlanSTIC de trois mois (mars à mai 2012) financé par la fondation Centrale Initiatives et le National Institute of Informatics,
portant sur l'inférence du modèle de Thomas sous-jacent à un Process Hitting.
Développement de l'outil \texttt{ph2thomas} permettant cette inférence à l'aide d'Answer Set Programming.
