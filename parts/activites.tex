\chapter{Activités scientifiques}

%\newcommand{\rev}{avec \textit{peer review}}

\newcommand{\ifnotempty}[2]{\ifthenelse{\equal{#1}{}}{}{#2}}
\newcommand{\replace}[3]{\ifthenelse{\equal{#1}{#2}}{#3}{#1}}

\newcommand{\entree}[4]{\begin{samepage}\vspace{.3em}\begin{itemize}\item[#1]\textbf{#2}\quad#3\end{itemize}\vspace{.3em}%
#4\vspace{.7em}\end{samepage}}

%\newcommand{\publi}[8]{\samepage{\vspace{.3em}\begin{itemize}\item[$\bullet$]\textbf{#1}\quad#2\end{itemize}\vspace{.3em}%
%\ifnotempty{#3}{#3. }%
%\ifnotempty{#4}{#4\ifnotempty{#5#6#7}{, }}%
%\ifnotempty{#6}{\replace{#5}{in}{in : }\textit{#6}\ifnotempty{#7}{, }}%
%#7\ifnotempty{#4#5#6#7}{.}\ifnotempty{#3#4#5#6#7}{\ifnotempty{#8}{\\}}%
%#8\ifnotempty{#3#4#5#6#7#8}{\vspace{.7em}}}}

\newcommand{\publi}[8]{\entree{$\bullet$}{#1}{#2}{%
\ifnotempty{#3}{#3. }%
\ifnotempty{#4}{#4\ifnotempty{#5#6#7}{, }}%
\ifnotempty{#6}{\replace{#5}{in}{in : }\textit{#6}\ifnotempty{#7}{, }}%
#7\ifnotempty{#4#5#6#7}{.}\ifnotempty{#3#4#5#6#7}{\ifnotempty{#8}{\\}}%
#8}}

\newcommand{\oldpubli}[8]{\entree{$\circ$}{#1}{#2}{%
\ifnotempty{#3}{#3. }%
\ifnotempty{#4}{#4\ifnotempty{#5#6#7}{, }}%
\ifnotempty{#6}{\replace{#5}{in}{in : }\textit{#6}\ifnotempty{#7}{, }}%
#7\ifnotempty{#4#5#6#7}{.}\ifnotempty{#3#4#5#6#7}{\ifnotempty{#8}{\\}}%
#8}}

\newcommand{\expose}[3]{\publi{#1}{}{#2}{}{}{}{#3}{}}

\newcommand{\oldexpose}[3]{\oldpubli{#1}{}{#2}{}{}{}{#3}{}}



$\bullet$ = Propre à l'année en cours (2012/2013)

$\circ$ = Propre à l'année précédente (2011/2012)

\section{Publications avec actes et sélection sur papier complet}
\label{sec:publications}

\subsection{En conférences internationales}

\publi
{CS2Bio'13}{accepté le \bemph{5 avril 2013} --- \cite{FPMR13-CS3Bio}}
{\bemph{Maxime Folschette}, Loïc Paulevé, Morgan Magnin, Olivier Roux}
{Under-approximation of reachability in multivalued asynchronous networks}
{in}{CS2Bio’13: 4th International Workshop on Interactions between Computer Science and Biology}
{Florence, Italie, Elsevier, juin 2013}
{Préimpression : \url{http://www.irccyn.ec-nantes.fr/~folschet/Folschette_CS2Bio13.pdf}}

\oldpubli
{CMSB'12}{accepté le 26 juin 2012 --- \cite{FPIMR12-CMSB}}
{\bemph{Maxime Folschette}, Loïc Paulevé, Katsumi Inoue, Morgan Magnin, Olivier Roux}
{Concretizing the Process Hitting into Biological Regulatory Networks}
{in}{CMSB'12: Proceedings of the 10th International Conference on Computational Methods in Systems Biology}
{Londres, Royaume-Uni, ACM, octobre 2012}
{\url{http://link.springer.com/chapter/10.1007\%2F978-3-642-33636-2_11}}

% Publi Nuclear Fusion
%\oldpubli
%{Précédente publication en journal}{dans le domaine de la fusion nucléaire}
%{Andrea Murari, Didier Mazon, Michela Gelfusa, \bemph{Maxime Folschette}, Thibaut Quilichini et collaborateurs EFDA-JET}
%{Residual analysis of the equilibrium reconstruction quality on JET}
%{}{Nuclear Fusion}
%{volume 51, numéro 5, avril 2011}%, DOI 10.1088/0029-5515/51/5/053012}
%{\url{http://iopscience.iop.org/0029-5515/51/5/053012/}}

\subsection{En workshop}

\oldpubli
{LDSSB'12}{accepté le 29 juillet 2012 --- \cite{FPIMR12-LDSSB}}
{\bemph{Maxime Folschette}, Loïc Paulevé, Katsumi Inoue, Morgan Magnin, Olivier Roux}
{Abducting Biological Regulatory Networks from Process Hitting models}
{in}{LDSSB'12: ECML/PKDD 2012 Workshop on Learning and Discovery in Symbolic Systems Biology}
{University of Bristol, Royaume-Uni, septembre 2012}
{Actes : \url{http://www.cs.bris.ac.uk/~oray/LDSSB12/LDSSB-2012.pdf}}



\section{Soumission en cours}
\label{sec:encours}

\publi
{TCS}{en phase de relecture finale}
{\bemph{Maxime Folschette}, Loïc Paulevé, Katsumi Inoue, Morgan Magnin, Olivier Roux}
{Constructing Biological Regulatory Networks from Process Hitting models}
{}{Theoretical Computer Science}{}
{Cet article a pour objectif d'être une version enrichie de~\cite{FPIMR12-CMSB}.
Il est déjà intégralement rédigé et est actuellement dans sa phase de relecture finale.}



\section{Séminaires}

\publi
{Résumé étendu et poster}{}
{\bemph{Maxime Folschette}}
{Presentation of the Process Hitting framework and inference of Biological Regulatory Networks with Thomas parameters}
{}{JDOC : 13\textsuperscript{e} Journée des Doctorants de l'ED STIM}
{Saint-Nazaire, France, \bemph{avril 2013}}
{Actes et posters : \url{https://sites.google.com/site/jdoc2013fr/}}

\publi
{Résumé étendu}{avec sélection}
{\bemph{Maxime Folschette}}
{Introduction to the Process Hitting and inference of its underlying Biological Regulatory Network}
{}{ASSB'13 : Thematic Research School on the Advances in Systems and Synthetic Biology}
{La Colle-sur-Loup, France, \bemph{mars 2013}}
{Préimpression : \url{http://www.irccyn.ec-nantes.fr/~folschet/Folschette_ASSB13.pdf}}

\publi
{Résumé étendu}{avec sélection}
{\bemph{Maxime Folschette}}
{Inferring Biological Regulatory Networks from Process Hitting models}
{}{MOVEP'12 : The 10th school for young researchers about Modelling and Verifying Parallel processes}
{Marseille, France, \bemph{décembre 2012}}
{Préimpression : \url{http://www.irccyn.ec-nantes.fr/~folschet/Folschette_MOVEP12.pdf}}



\section{Exposés}

\oldexpose{Réunion du groupe de travail ANR BioTempo/G2 : « Des dynamiques discrètes à des dynamiques continues (modèles hybrides) »}
{Concretizing Process Hitting models into Biological Regulatory Networks with Thomas' formalism using ASP}
{Juin 2012, Nantes, France}

\oldexpose{Séminaire informel AED}
{Modeling and Analysis of Large Biological Regulatory Networks thanks to the “Process Hitting” Framework}
{Juin 2012, Nantes, France}

\oldexpose{Fourth CSPSAT \& ASP Seminar}
{Concretizing Process Hitting models into Biological Regulatory Networks with Thomas' formalism using ASP}
{Mai 2012, Kobe, Japon}

\oldexpose{KUBIC-NII Joint Seminar on Bioinformatics 2012}
{Translating Process Hitting models to Thomas' modeling with ASP}
{Avril 2012, Kyoto, Japon}

\oldexpose{The 8th Meeting on Inference-based Hypothesis-finding and its Application to Systems Biology}
{Modeling and Analysis of Large Biological Regulatory Networks thanks to the Process Hitting Framework}
{Mars 2012, Kanazawa, Japon}



\section{Collaboration}
\label{sec:collaboration}

\entree{$\bullet$}
{Katsumi Inoue}{Inoue Lab., National Institute of Informatics, Tokyo, Japon}
{Collaboration nouée lors du stage doctoral à l'Inoue Lab. en première année (mars à mai 2012).
Co-auteur de la publication acceptée à CMSB'12~\cite{FPIMR12-CMSB} et
participation à la révision de ce même article pour sa soumission à Theoretical Computer Science (cf.~sections~\ref{sec:tcs} et~\ref{sec:encours}).}


