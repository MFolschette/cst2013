\chapter{Contributions}

\section{Inférence de paramètres par la logique de Hoare \normalsize(en cours)}
\label{sec:hoare}
\todo{À laisser ?}

Une thèse de master \cite{Folschette2011} dont le sujet était “Application de la logique de Hoare aux Réseaux de Régulation Génétiques avec multiplexes” a été réalisée au sein de la même équipe avant le début de ce doctorat.
Son objectif était d'étudier des travaux en cours permettant de déduire les propriétés nécessaires d'un système pour observer certains comportements dynamiques, et d'implémenter une telle analyse.
La logique de Hoare, introduite dans \cite{hoare-69}, se base sur un triplet (dit \emph{triplet de Hoare}) qui représente :
\begin{itemize}
  \item une pré-condition (état initial),
  \item un programme impératif (chemin d'exécution),
  \item une post-condition (état final).
\end{itemize}
Un tel triplet signifie que, partant d'un état initial respectant la pré-condition, l'exécution du programme amènera toujours le système dans un état final respectant la post-condition.
D'après les travaux de Dijkstra~\cite{dijkstra-75}, il est possible, en ne connaissant que le programme et la post-condition,
d'inférer la plus faible pré-condition possible permettant d'obtenir un tel triplet.

Ainsi, les travaux sur lesquels se basent la thèse de master utilisent la logique de Hoare pour,
à partir d'un ensemble de dynamiques possibles d'un système biologique (représentées par un programme impératif)
et d'un ensemble d'états finals du système (représentés par une post-condition),
obtenir la plus large condition nécessaire le système, par calcul de la plus faible pré-condition.
Cette méthode se distingue par la possibilité d'inférer, en plus d'un ensemble d'états initiaux, une classe de paramétrisations pour lesquelles les dynamiques données sont possibles.
De plus, sa complexité est fortement réduite par rapport à un calcul du Graphe des États.
Cependant, son aboutissement lors du travail de master n'avait été que partiel face à certaines difficultés d'implémentation, bien que l'essai eut été effectué avec trois approches différentes :
\begin{itemize}
  \item une approche mathématique sous forme de preuves avec Coq,
  \item une approche fonctionnelle en OCaml,
  \item une approche logique en Prolog.
\end{itemize}

Fort d'un recul supplémentaire après un an de doctorat et de la connaissance d'ASP acquise au National Institute of Informatics, j'ai pu me replonger dans ce problème afin de proposer une approche et une implémentation efficaces.
Cette nouvelle implémentation se base sur une représentation arborescente (et non fonctionnelle) des propriétés (pré-condition et post-condition) ce qui permet, contrairement aux implémentations précédentes, de les manipuler aisément.
Cette nouvelle approche se compose de trois modules aux objectifs distincts :
\begin{itemize}
  \item Le noyau est écrit en OCaml et permet de calculer la plus faible pré-condition à partir du couple programme et post-condition donné ;
  \item L'énumération des solutions possibles (état initial et paramétrisation) peut être réalisé à l'aide d'un programme ASP représentant la pré-condition obtenue ;
  \item La fonction de simplification, ajoutée au noyaux en OCaml, permet enfin de simplifier la pré-condition obtenue lorsque certains termes sont trivialement vrais ou faux, ou l'affaiblir en fonction de contraintes sur les paramètres et la condition initiale ; cette simplification s'avère utile pour manipuler la propriété obtenue sans avoir à calculer explicitement la classe de solutions sous-jacente.
\end{itemize}

Cette nouvelle implémentation n'a pas encore fait l'objet d'une publication ou d'une communication, faute de la publication de l'article théorique initial.
Cependant, elle est fonctionnelle et les résultats sont satisfaisants car elle permet de répondre aux divers exemples proposés ainsi qu'à des cas appliqués plus complexes.
Elle permet actuellement de répondre en moins d'une seconde \todo{chiffres précis} là où les anciennes implémentations sont fortement limitées :
\begin{itemize}
  \item en temps lorsqu'il faut calculer le Graphe des États pour chaque paramétrisation possible,
  \item en mémoire lorsqu'il faut énumérer explicitement tous les états initiaux et paramétrisations en même temps.
\end{itemize}
En ce sens, ce travail s'intègre parfaitement à la problématique d'exploitation des grands Réseaux de Régulation Biologiques.



\section{Analyse d'atteignabilité dans un réseau discret \normalsize(publié)}
\label{sec:cs2bio}

Parmi les perspectives de l'année précédente figurait une « réflexion sur les propriétés d'atteignabilité », notamment par l'enrichissement de la sémantique du Process Hitting et l'adaptation de l'analyse statique.
En effet, l'ajout d'expressivité à la sémantique de base du Process Hitting, et notamment de priorités fixes entre les actions, rend caduque certaines propriétés d'atteignabilité, car la dynamique évolue en conséquence.

L'objectif d'une extension de sémantique est la création de modèles plus justes et plus proches de modélisations existantes.
Ainsi, la sémantique de base du Process Hitting permet une coopération entre composants à l'aide de sortes particulières appelées sortes coopératives.
L'objectif d'une sorte coopérative est de représenter chacun des états entremêlés de plusieurs composants.
Par exemple, la sorte $ab$ de la figure~\ref{fig:exPH} est une sorte coopérative sur les sortes $a$ et $b$ : chacun de ses processus représente un sous-état des états indépendants de $a$ et $b$ ($ab_{01}$ modélise la présence de $a_0$ et $b_1$, etc.).
Les actions frappant la sorte $ab$ permettent de \emph{mettre à jour} l'état de cette sorte en fonction de l'état des composants qu'elle représente ;
par exemple, les actions $\PHfrappe{a_0}{ab_{11}}{ab_{01}}$ et $\PHfrappe{a_0}{ab_{10}}{ab_{00}}$ permettent sa mise à jour lorsque le processus $a_0$ est actif.
Ainsi, l'accessibilité d'un processus de la sorte coopérative est conditionnée par l'accessibilité de chaque processus qu'il représente.
Il est alors possible d'ajouter une action déclenchée par un processus de la sorte coopérative plutôt que plusieurs actions indépendantes partant des composants qui coopèrent, comme l'action $\PHfrappe{ab_{11}}{c_0}{c_1}$ de la figure~\ref{fig:exPH}.

Un tel mécanisme à base de sortes coopératives souffre cependant d'un problème de “décalage temporel” au niveau des mises à jour.
En effet, rien de force à jouer prioritairement les actions de mise à jour des sortes coopératives par rapport aux autres actions.
Par conséquent, une sorte coopérative représente toujours l'état présent ou un état passé du système, ce autorise parfois des comportements non souhaitables.
Par exemple, dans le Process Hitting de la figure~\ref{fig:exPH}, la dynamique suivante est possible (les processus qui évoluent sont indiqués en gras) :
$$
  \PHetat{a_1, b_0, ab_{10}, c_0} \rightarrow
  \PHetat{\mathitbf{a_0}, b_0, ab_{10}, c_0} \rightarrow
  \PHetat{a_0, \mathitbf{b_1}, ab_{10}, c_0} \rightarrow
  \PHetat{a_0, b_1, \mathitbf{ab_{11}}, c_0} \rightarrow
  \PHetat{a_0, b_1, ab_{11}, \mathitbf{c_1}}
$$
En d'autres termes, le processus $ab_{11}$ peut être atteint alors que les processus $a_1$ et $b_1$ n'ont pas été présents simultanément (\ie au sein du même état).
On peut d'ailleurs montrer que si l'état initial ne contient pas $a_1$ et $b_1$, alors ceux-ci ne seront jamais présents simultanément.
Un tel comportement est possible car la mise à jour de la sorte coopérative $ab$ n'est pas forcée :
$ab_{10}$ peut rester actif même lorsque $a_1$ n'est plus actif.

Afin de pallier cela, la notion de priorités fixes avait déjà été évoquée.
Cette idée a été développée et a donné lieu à une publication retenue au workshop CS2Bio 2013 \cite{FPMR13-CS3Bio}.
Les apports techniques de ce travail se divisent en deux point principaux.
\begin{itemize}
  \item D'une part, une extension formelle du Process Hitting est proposée, sous la forme d'une sémantique avec mise à jour prioritaire des sortes coopératives.
  Ces priorités permettent de résoudre les problèmes de décalage temporel vus précédemment.
  Elles sont restreintes dans ce travail à l'ajout d'une seule classe d'actions priorisées, à laquelle appartiennent toutes les actions de mise à jour des sortes coopératives et qui doivent ainsi être jouées de façon prioritaire, avant toute autre action non priorisée,
  Cette classe d'actions priorisées est représentée dans l'exemple de la figure~\ref{fig:exPH} par l'ensemble des actions en trait épais :
  l'accessibilité de $ab_{11}$ est rendue impossible car lorsque le processus $a_0$ est atteint, l'action prioritaire $\PHfrappe{a_0}{ab_{10}}{ab_{00}}$ doit nécessairement être jouée avant l'action $\PHfrappe{a_0}{b_0}{b_1}$.
  \item D'autre part, il a été nécessaire d'adapter une partie des méthodes initiales d'analyse statique qui ne s'appliquent pas à cette sémantique étendue.
  La méthode de sous-approximation a été revue et prend désormais en compte le caractère priorisé des actions de mise à jour.
  La méthode de sur-approximation reste en revanche valable sur le Process Hitting “aplati” (\ie sans considérer les classes de priorité).
\end{itemize}
Il a enfin été montré que cette sémantique étendue est bisimilaire aux réseaux discrets (dont fait partie la modélisation de Thomas).
Ainsi, nous avons développé une méthode d'analyse d'atteignabilité très efficace sur les réseaux discrets booléens et multivalués.
Ces travaux ont été intégrés à la bibliothèque Pint et testés notamment sur un modèle booléen de 94 composants :
les résultats sont tous conclusifs et ont été obtenus en quelques centièmes de secondes sur un ordinateur personnel,
soit une performance bien supérieure à celle d'un model-checker standard.

Ces résultats font écho à la première étude réalisée au début de la thèse, qui avait abouti à la formalisation de deux nouvelles sémantiques plus expressives du Process Hitting.
La première introduisait la notion d'\emph{arc neutralisant} modélisant une priorité locale entre deux actions ;
il a été montré que cette sémantique est faiblement bisimilaire à la sémantique du Process Hitting avec priorités fixes.
La seconde consistait en la généralisation de la notion d'action à celle d'\emph{action conjointe} possédant plusieurs frappeurs ;
cette sémantique est faiblement bisimilaire à la sémantique présentée dans cette section car une action conjointe possède le même comportement qu'une sorte coopérative avec actions priorisées bien choisie.



\section{Extraction de Réseaux de Régulation Biologiques complets depuis un modèle de Process Hitting \normalsize(en cours)}
\label{sec:tcs}

Cette première année de thèse avait été l'occasion de compléter les liens formels entre modèle de Thomas et Process Hitting,
en publiant notamment une méthode de traduction d'un modèle en Process Hitting vers un (ensemble de) modèle(s) de Thomas dont le comportement est strictement inclus.
Ce travail avait été réalisé lors d'un stage doctoral de trois mois au National Institute of Informatics à Tokyo, supervisé par le professeur Katsumi Inoue, et avait abouti à une implémentation sous la forme de l'outil \texttt{ph2thomas} intégré à la bibliothèque existante Pint\footnote{Disponible à \url{http://process.hitting.free.fr}} et utilisant notamment un paradigme de programmation logique appelé \emph{Answer Set Programming} (ASP).
Il a fait l'objet d'une publication en workshop présentée en septembre \cite{FPIMR12-LDSSB}, d'une publication en conférence internationale \cite{FPIMR12-CMSB} présentée en octobre, et donne lieu aujourd'hui à une collaboration durable entre l'équipe MeForBio et l'Inoue Laboratory.
Ce travail fait actuellement l'objet d'une révision en vue de sa soumission dans la revue Theoretical Computer Science.

L'inférence d'un modèle de Thomas depuis un Process Hitting donné se décompose en deux étapes principales :
\begin{itemize}
  \item L'inférence du Graphe des Interactions (GI) permet d'obtenir la structure du modèle et des interactions entre composants.
  Elle doit être réalisée en premier car la paramétrisation du modèle dépend du GI.
  \item L'inférence des paramètres permet d'obtenir une paramétrisation (possiblement partielle) telle que le modèle final respecte la dynamique du Process Hitting initial (aucun comportement supplémentaire n'est possible).
\end{itemize}
Ces deux étapes peuvent échouer car certains comportement du Process Hitting ne peuvent être représentés pas un modèle de Thomas.
Ainsi, si certains paramètres n'ont pas pu être inférés lors de la seconde étape, cela signifie généralement que la dynamique du Process Hitting est trop générale pour être représentée par un unique modèle de Thomas.
Dans ce cas, une énumération des paramétrisations complètes et compatibles peut être effectuée pour rattraper cet échec, afin d'obtenir une classe de modèles de Thomas respectant la dynamique du modèle initial.
De même, la première étape de l'inférence peut, en outre des arcs positifs et négatifs du GI, produire des arcs non-signés lorsque certaines régulations ne peuvent être résumés à de simples activations ou inhibitions.
Cependant, dans la première version de ce travail, cette première étape ne pouvait être rattrapée,
car un tel arc non-signé ne pouvait être exploité par l'étape suivante, faisant alors échouer l'inférence à mi-parcours.

L'un des objectifs de l'enrichissement de ce travail a été de pouvoir exploiter les inférences d'arcs non-signés.
Pour cela, nous proposons une sémantique étendue du modèle de Thomas, comprenant des arcs positifs ($+$), négatifs ($-$) et non-signés ($\uns$).
Cette extension est possible du fait que le signe des arcs n'impacte pas la dynamique du modèle ; seule la paramétrisation spécifie la direction d'évolution du modèle dans chaque état.
Cependant, de la même façon qu'un arc positif (resp.~négatif) modélise sans ambiguïté une activation (resp.~une inhibition),
un arc non-signé modélise une régulation donc la nature est indéterminée ou plus complexe qu'une simple activation ou inhibition.
Étant donné ce nouvel outil, l'inférence d'un GI est possible dans tous les cas de figure, et un seuil est inféré quel que soit le signe.
L'inférence des paramètres est alors possible malgré la présence d'arcs non-signés étant donné que ceux-ci n'influent pas sur la dynamique.
Enfin, l'énumération des paramétrisations compatibles a aussi été revue afin d'être adaptée à cette nouvelle sémantique.
La compatibilité est assurée par des contraintes supplémentaires qui assurent notamment qu'un paramètre est utile (hypothèse d'activité) et cohérent avec le type de la régulation (hypothèse de monotonicité), en plus de respecter la dynamique du Process Hitting.
Le signe des arcs entre ainsi en compte dans cette énumération ;
cependant, les arcs non-signés n'apportent pas de contrainte à ce niveau afin de permettre toutes les réponses possibles, étant donné que leur nature exacte est indéterminée.
L'implémentation précédemment réalisée dans le cadre de ce travail a été adaptée, rendant l'inférence conclusive dans un plus grand nombre de cas.
Les résultats sont toujours produits en moins d'une seconde sur un ordinateur de bureau pour des modèles de 20 et 40 gènes.

Cette révision est aussi l'occasion d'enrichir la rédaction de discussions et d'explications supplémentaires à propos de certains points n'ayant pas été abordés dans la première version, faut de place.
