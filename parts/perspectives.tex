\chapter{Perspectives}

\todo{Revoir !!!!
\begin{itemize}
  \item Priorités multiples avec précédence
  \item Traduction CIF2PH (Carito)
  \item Multiplexes avec délais (Morgan)
\end{itemize}
}



\section{Données chronométriques}
À partir de modèles algébriques tels que le Process Hitting ou le modèle de Thomas, il est possible d'extraire des propriétés à caractère chronologique,
par l'étude du Graphe des États ou à l'aide d'analyses statiques permettant de conclure quant à des atteignabilités successives de processus.
Cependant, il peut être nécessaire de contraindre ou d'étudier un système en fonction de données chronométriques, portant sur la durée des processus mis en jeu plutôt que sur leur simple succession.
L'une des perspectives de ce sujet serait ainsi d'étudier les possibilités d'étendre la sémantique du Process Hitting afin de pouvoir y inclure des données chronométriques discrètes ou continues.

%Une telle extension pourrait revêtir différents buts.
L'un des principaux buts d'une telle extension est l'inférence des délais au sein d'un modèle afin d'obtenir des résultats temporels sur certains phénomènes difficilement reproductibles ou observables expérimentalement.
L'étude du cycle circadien, qui fait partie des thématiques étudiées par MeForBio \textit{via} notamment le projet CirClock, est un exemple de domaine qui bénéficierait directement d'un tel développement.
D'autres objectifs peuvent être dégagés de cette extension, comme l'ajout d'un système d'horloges à la modélisation permettant la représentation de comportements plus complexes et dépendant du temps.

Cette perspective s'inscrit dans la collaboration en cours de développement avec l'équipe du professeur Katsumi Inoue du National Institue of Informatics,
dont le but est de développer des outils permettant l'inférence de données chronométriques à l'aide d'outils SAT.
%dans le cadre d'un projet STIC-Asie SATTIBIS

%Projet avec le Japon
%+ sujet doctorat



\section{Réflexion sur les propriétés d'atteignabilité}

L'une des forces principales du Process Hitting repose en les outils d'analyse statique permettant de vérifier des propriétés d'atteignabilité.
Cependant, ces outils s'appliquent uniquement à la sémantique de base du Process Hitting ;
les ajouts à la sémantique entraînent généralement l'impossibilité d'utiliser ces outils.
Une perspective intéressante serait ainsi de mener une réflexion approfondie sur l'application de ces outils à d'autres sémantiques.

Le Process Hitting à priorités fixes, au moins dans certaines restrictions de son utilisation, pourrait notamment bénéficier de ces outils.
Leur application aux autres sémantiques présentées à la section \todo{xx} pourrait pourrait être intéressante sous réserve que ces sémantiques présentent un intérêt pour certaines modélisation.
Enfin, cette réflexion devra être menée durant tout le processus de développement d'une sémantique de Process Hitting intégrant des données chronométriques,
afin d'en permettre une exploitation efficace.
