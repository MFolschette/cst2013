\documentclass[11pt]{report}

\usepackage{hyperref}

\usepackage[french]{babel}
\usepackage[utf8]{inputenc}
\usepackage[T1]{fontenc}

\usepackage{amsmath}  % Maths
\usepackage{amsfonts} % Maths
\usepackage{amssymb}  % Maths
\usepackage{stmaryrd} % Maths (crochets doubles)

\usepackage{url}     % Mise en forme + liens pour URLs
\usepackage{array}   % Tableaux évolués

\usepackage{comment}

\usepackage[hmargin=3cm,vmargin=2.5cm]{geometry}

\usepackage[font=footnotesize]{caption}  % Style des légendes

\usepackage{tikz}
\newdimen\pgfex
\newdimen\pgfem
\usetikzlibrary{arrows,shapes,shadows,scopes}
\usetikzlibrary{positioning}
\usetikzlibrary{matrix}
\usetikzlibrary{decorations.text}
\usetikzlibrary{decorations.pathmorphing}

\input{macros/macros}
\input{macros/macros-ph}
\input{macros/macros-grn}

% Commandes À FAIRE
\usepackage{color} % Couleurs du texte
%\definecolor{darkgreen}{rgb}{0,0.5,0}
%\newcommand{\afaire}[1]{\textcolor{red}{[À FAIRE~: #1]}}
%\newcommand{\resume}[1]{\textcolor{blue}{#1}}
\newcommand{\todo}[1]{\textcolor{red}{\textbf{[TODO : #1]}}}


% Id est
\newcommand{\ie}{\textit{i.e.}~}

% Maths en gras et italique
\DeclareMathAlphabet{\mathitbf}{OML}{cmm}{b}{it}

% Symbole « non-signé »
\newcommand{\uns}{\pm}

% Mise en valeur
\newcommand{\bemph}[1]{\textbf{#1}}






\title{Rapport pour le comité de suivi de thèse 2012/2013}

\author{Maxime Folschette}

%LUNAM Universit\'e, \'Ecole Centrale de Nantes, IRCCyN UMR CNRS 6597\\
%(Institut de Recherche en Communications et Cybern\'etique de Nantes)\\
%1 rue de la No\"e -- B.P. 92101 -- 44321 Nantes Cedex 3, France.\\
%\email{Maxime.Folschette@irccyn.ec-nantes.fr}

\begin{document}

%\maketitle


\thispagestyle{empty}

~
%\vspace{5cm}
\vfill
{\Huge
\begin{center}
\textbf{Rapport pour le comité de suivi\\de thèse 2012/2013}

\vspace{1cm}

\LARGE
Modélisation algébrique de l'évolution et de la dynamique multi-échelles des réseaux de régulation biologique

\vspace{2cm}

\normalsize
20 juin 2013

\vspace{1cm}

\Large
Maxime Folschette

\vspace{2cm}

\normalsize
\begin{tabular}{rl}
  Directeur :&Olivier Roux\\
  Co-encadrant :&Morgan Magnin\\
  &\\
  &\\
  Établissement :&École Centrale de Nantes\\
  Laboratoire :&IRCCyN, UMR CNRS 6597\\
  Équipe :&MeForBio\\
  École doctorale :&STIM
\end{tabular}

\end{center}

\vspace{2cm}
}
%\vfill

\setcounter{page}{0}
\tableofcontents


\chapter{Problématique}

\todo{Réécrire ?}

L'étude et la compréhension des mécanismes du vivant, notamment au sein de la machine cellulaire, posent de nombreux problèmes de représentation.
Plusieurs critères participent à la qualité d'un modèle biologique, et influent sur sa facilité d'élaboration, de lecture et d'exploitation.
Le formalisme utilisé doit en effet permettre de représenter de façon pratique et complète un processus biologique donné, et donc
de proposer une sémantique claire, sans lacune et compréhensible par une large communauté.
De plus, son exploitation doit dans l'idéal être efficace, c'est-à-dire permettre l'obtention de résultats en évitant les problèmes de complexité en mémoire et en temps.

Dans ce cadre, les Réseaux de Régulation Biologiques permettent de représenter des systèmes biologiques souvent déterminés par les physiciens en termes d'équations différentielles.
Ces équations étant souvent difficiles à résoudre, elles sont simplifiées sous la forme de systèmes algébriques dont les influences entre composants se résument à des activations et des inhibitions.

\section{Modèle de Thomas}
Le modèle de Thomas, introduit en 1973 dans \cite{Thomas73}, propose une simplification cohérente des modèles continus sous forme d'équations différentielles.
Plutôt que de considérer les valeurs réelles de concentration des protéines synthétisées, ce modèle repose sur l'utilisation de seuils qui représentent des valeurs particulières de ces concentrations au-delà desquelles les influences entre composants évoluent.
En plus de proposer une approche qualitative cohérente, cette représentation permet de s'affranchir de la connaissance des valeurs réelles des concentrations, qui sont souvent difficiles à obtenir expérimentalement.

Il est possible de représenter simplement un système dans ce formalisme sous la forme d'un Graphe des Interactions dont les nœuds représentent les composants et les arcs étiquetés et orientés représentent les interactions entre ces composants.
Ce modèle introduit également la notion de paramètre discret, qui permet de spécifier la dynamique du système, notamment dans les cas de coopérations entre composants.
De tels paramètres jouent le rôle de points focaux dans le sens où il déterminent la direction d'évolution du système dans chacune de ses configurations.
La figure \ref{fig:exRRB} donne un exemple de Réseau de Régulation Biologique simple.

\begin{figure}[ht]
\begin{minipage}{0.4\linewidth}
\centering
\scalebox{1.2}{
\begin{tikzpicture}[grn]
\path[use as bounding box] (0,-0.7) rectangle (3.5,0.7);
\node[inner sep=0] (a) at (2,0) {a};
\node[inner sep=0] (b) at (0,0) {b};
\node[inner sep=0] (c) at (3.5,0) {c};
%\path
%  node[elabel, below=-1em of a] {$0..2$}
%  node[elabel, below=-1em of b] {$0..1$}
%  node[elabel, below=-1em of c] {$0..1$};
\path[->]
  (b) edge[bend right] node[elabel, below=-3pt] {$+1$} (a)
  (c) edge node[elabel, above=-5pt] {$+1$} (a)
  (a) edge[bend right] node[elabel, above=-5pt] {$-2$} (b);
\end{tikzpicture}
}
\end{minipage}
\begin{minipage}{0.6\linewidth}
\centering
\begin{align*}
K_{a,\{b,c\}} &= 2 & K_{b,\emptyset} &= 1 \\
K_{a,\{b\}} &= 1 & K_{b,\{a\}} &= 0 \\
K_{a,\{c\}} &= 1 &&\\
K_{a,\emptyset} &= 0 & K_{c,\emptyset} &= 1
\end{align*}
\end{minipage}
\caption{\label{fig:exRRB}
Exemple de Réseau de Régulation Biologique, comprenant un Graphe des Interactions (à gauche) et une paramétrisation (à droite).
Chaque nœud du graphe représente un composant et chaque arc une régulation
dont l'étiquette indique le type (“$+$” pour une activation et “$-$” pour une inhibition) et le seuil.
Les paramètres tiennent lieu de points focaux pour l'état concerné.
}
\end{figure}

Aujourd'hui plus largement utilisé sous sa forme multivaluée \cite{richard-comet-bernot-08}, le modèle de Thomas a connu un certain nombre d'extensions,
comme l'ajout de multiplexes \cite{bernot-comet-khalis-08},
l'intégration du temps continu \cite{Ahmad08},
ou encore l'étude de sémantiques plus expressives \cite{BernotSemBRN}.
Il a aussi été l'objet de travaux concernant la prédiction de son comportement d'après le Graphe des Interactions \cite{RiCo07},

Bien qu'il soit adapté à la représentation des Réseaux de Régulation Biologiques, et que son utilisation soit très répandue, le modèle de Thomas souffre de deux principaux inconvénients.
Tout d'abord, son utilisation pour la recherche de propriétés intéressantes sur les systèmes modélisés nécessite souvent l'analyse du Graphe des États dont le calcul s'avère être de complexité exponentielle.
De plus, l'étude d'un système représenté par ce formalisme nécessite une parfaite connaissance des coopérations entre composants, et donc d'avoir choisi une paramétrisation complète au sein d'un ensemble de possibilités potentiellement très important.



\section{Process Hitting}\label{sec:PH}
Une modélisation des Réseaux de Régulation Biologiques à l'aide de $\pi$-calcul, appelée Process Hitting (ou Frappes de Processus), a été récemment introduite par l'équipe MeForBio \cite{PMR10-TCSB,PaulevePhD} et constitue le sujet central de cette thèse.
Elle propose un point de vue plus modulable des influences entre composants grâce à une représentation d'actions atomiques entre ceux-ci.
Cette représentation particulière offre des possibilités d'analyse statique efficaces permettant de sur-approximer et sous-approximer l'atteignabilité d'un processus \cite{PMR12-MSCS}.
De plus, son atomicité permet d'adopter différents niveaux d'abstraction dans la modélisation, afin notamment de représenter une sur-approximation du comportement d'un système dont la spécification des coopérations ne serait pas entièrement déterminée.
Une méthode efficace de détermination des points fixes a aussi été développée.
La figure \ref{fig:exPH} donne un exemple de Process Hitting.

Plusieurs extensions ont aussi été proposées pour enrichir ce formalisme.
Une première repose sur l'introduction de stochasticité afin de modéliser la durée d'évolution relative des composants à l'aide probabilités.
Cette extension nécessite l'exécution du modèle afin d'en extraite des propriétés empiriques, ou l'utilisation d'un model checker.
Une seconde extension consiste en l'attribution de classes de priorités aux actions, afin d'imposer formellement un ordre de tir entre celles-ci.
Cette sémantique reposant sur des priorités fixes permet de modéliser des comportements plus fins, par exemple au niveau des coopérations.
Elle ne modifie pas les résultats concernant la recherche de points fixes, mais n'est pas compatible avec la méthode de sous-approximation de l'analyse statique.

\begin{figure}[ht]
\centering
\scalebox{1.1}{
\begin{tikzpicture}
\path[use as bounding box] (-2,-5.2) rectangle (7,0.7);

\TSort{(0,0)}{b}{2}{t}
\TSort{(0,-3.8)}{c}{2}{b}
\TSort{(4.5,-3)}{a}{3}{r}

\TSetTick{bc}{0}{00}
\TSetTick{bc}{1}{01}
\TSetTick{bc}{2}{10}
\TSetTick{bc}{3}{11}
% \TSetSortLbcel{bc}{$\neg a\wedge b$}
\TSort{(-0.5,-2)}{bc}{4}{b}

\THit{b_1}{bend right}{bc_0}{.north}{bc_2}
\THit{b_1}{bend right}{bc_1}{.north}{bc_3}
\THit{b_0}{}{bc_2}{.north west}{bc_0}
\THit{b_0}{}{bc_3}{.north west}{bc_1}

\THit{c_0}{}{bc_1}{.south}{bc_0}
\THit{c_0}{}{bc_3}{.south}{bc_2}
\THit{c_1}{}{bc_0}{.south}{bc_1}
\THit{c_1}{}{bc_2}{.south}{bc_3}

\path[bounce, bend right=25]
\TBounce{bc_2}{}{bc_0}{.north east}
\TBounce{bc_3}{}{bc_1}{.north east}
;
\path[bounce, bend left=80, distance=30]
\TBounce{bc_0}{}{bc_2}{.north}
\TBounce{bc_1}{}{bc_3}{.north}
;
\path[bounce, bend right]
\TBounce{bc_0}{}{bc_1}{.west}
\TBounce{bc_2}{}{bc_3}{.west}
;
\path[bounce, bend left]
\TBounce{bc_3}{}{bc_2}{.east}
\TBounce{bc_1}{}{bc_0}{.east}
;

\THit{bc_3}{thick}{a_1}{.north west}{a_2}
\THit{bc_0}{thick,bend right=130, in=305, distance=140}{a_1}{.south east}{a_0}
\path[bounce, bend left=40]
\TBounce{a_1}{thick}{a_2}{.south west}
\TBounce{a_1}{thick}{a_0}{.north east}
;

\THit{b_0}{thick,bend left,out=50,in=150}{a_2}{.west}{a_1}
\THit{b_1}{thick,bend left,out=80,in=70,distance=100}{a_0}{.east}{a_1}
\path[bounce]
\TBounce{a_2}{thick,bend right=40}{a_1}{.west}
\TBounce{a_0}{thick,bend right=40}{a_1}{.east}
;

\THit{c_0}{thick,bend left,out=270,in=290, distance=115}{a_2}{.east}{a_1}
\THit{c_1}{thick}{a_0}{.north west}{a_1}
\path[bounce]
\TBounce{a_2}{thick,bend left=40}{a_1}{.north east}
\TBounce{a_0}{thick,bend left=40}{a_1}{.south west}
;

\THit{a_2}{bend left, out=290, in=120}{b_1}{.south}{b_0}
\path[bounce, bend left]
\TBounce{b_1}{}{b_0}{.south}
;

\end{tikzpicture}
}

\caption{\label{fig:exPH}
Exemple de Process Hitting comprenant quatre sortes : $a$, $b$, $c$ et $bc$.
La sorte $bc$ est appelée sorte coopérative (cf. section \ref{sec:cs2bio}).
}
\end{figure}



\section{Enjeux}

\todo{À revoir}

Le développement et l'enrichissement de nouveaux formalismes possède comme but premier une meilleure représentation des systèmes et des phénomènes modélisés.
L'un des objectifs de l'équipe MeForBio est l'étude formelle des propriétés de systèmes de grande taille,
en s'appuyant notamment sur des représentation complémentaires et adaptées de ceux-ci.

La thèse de master \cite{Folschette2011} réalisée dans la même équipe avait consisté en une introduction à la recherche de telles propriétés formelles sur le modèle de Thomas en utilisant la logique de Hoare.
Cependant, son aboutissement n'avait été que partiel face à certaines difficultés d'implémentation,
et d'autres pistes concernant cette fois le Process Hitting, qui permet notamment l'analyse de grands systèmes, peuvent être envisagées au cours de la présente thèse de doctorat.
De plus, l'étude de propriétés formelles pourra s'accompagner d'un enrichissement de la sémantique du formalisme afin d'aborder des aspects qui ne sont actuellement pas pris en compte.
Une possibilité serait un travail sur l'introduction progressive du temps dans le modèle pour une analyse chronométrique des systèmes biologiques.
%comme des calculs de délais à l'aide de temps chronométrique.

%L'une d'elles pourra consister en l'extension des outils déjà développés à d'autres sémantiques du Process Hitting afin 
%La thématique de l'ajout de propriétés temporelles à ce modèle pourra notamment être reprise, dans le but de quantifier un phénomène non plus par une simple chronologie d'événements mais aussi par des données quantitatives concernant des délais d'évolution.
%De telles données chronométriques permettraient de généraliser les propriétés et outils développés sur la sémantique du Process Hitting standard.

Un autre enjeu de la présente thèse consistera en la compréhension des liens entre les différents formalismes utilisés.
Cette compréhension est nécessaire afin de dégager efficacement les atouts de chaque type de modèle,
mais aussi dans le but d'assurer une bonne communication au sein de la communauté scientifique, qui ne partage pas toujours les mêmes représentations.
Une contribution a déjà été apportée sur ce plan dans \cite{FPIMR12-CMSB} (cf. section \ref{sec:traduction}).

Tous ces aspects ont pour but commun le développement de formalismes ayant des propriétés intéressantes ainsi que des outils adéquats.
Ces formalismes doivent en effet pouvoir modéliser efficacement les systèmes étudiés,
en faisant face de façon efficace aux problèmes de complexité qui se posent lors de l'étude de grands modèles.

\chapter{Contributions}

\section{Retour sur les contributions passées}\label{sec:retour}

Un travail préliminaire à cette thèse a consisté en la formalisation de deux nouvelles sémantiques du Process Hitting plus expressives.
La première introduit la notion d'\emph{arc neutralisant} qui modélise une priorité locale entre deux actions ;
il a été montré que cette sémantique est faiblement bisimilaire à la sémantique du Process Hitting avec priorités fixes.
Une seconde consiste en la généralisation de la notion d'action à celle d'\emph{action conjointe}, possédant plusieurs frappeurs ;
cette sémantique est identique à celle du Process Hitting à 2 priorités fixes et mise à jour prioritaire des sortes coopératives, comme détaillé dans la section~\ref{sec:cs2bio}.

De plus, cette première année de thèse avait été l'occasion de compléter les liens formels entre modèle de Thomas et Process Hitting,
en mettant notamment à jour une méthode de traduction d'un Process Hitting en un (ensemble de) modèle(s) de Thomas dont le comportement est strictement inclus.
Ce travail a été réalisé lors d'un stage doctoral de trois mois au National Institute of Informatics à Tokyo, supervisé par le professeur Katsumi Inoue, et a abouti à une implémentation sous la forme de l'outil \texttt{ph2thomas} intégré à la bibliothèque existante Pint\footnote{Disponible à \url{http://process.hitting.free.fr}} et utilisant notamment une forme de programmation logique appelée \emph{Answer Set Programming} (ASP).
Il fait l'objet d'une publication en workshop présentée en septembre \cite{FPIMR12-LDSSB}, d'une publication en conférence internationale \cite{FPIMR12-CMSB} présentée en octobre, et a donné lieu à une collaboration durable entre l'équipe MeForBio et l'Inoue Lab.
\todo{Il fait aussi actuellement l'objet d'une extension en vue de sa soumission en revue.}

\section{Implémentation Hoare}
\todo{À faire ?}

Une thèse de master \cite{Folschette2011} dont le sujet était la recherche de propriétés formelles sur le modèle de Thomas en utilisant la logique de Hoare et été réalisée au sein de la même équipe avant le début de ce doctorat.
L'objectif était d'étudier des travaux en cours permettant de déduire les propriétés nécessaires d'un système pour observer certains comportements dynamiques, et d'implémenter une telle analyse.
L'approche par logique de Hoare se base sur des triplets de Hoare qui représentent :
\begin{itemize}
  \item une pré-condition (état initial),
  \item un programme impératif (représentant un ensemble de dynamiques possibles),
  \item une post-condition (état final).
\end{itemize}
Cette approche en particulier utilise la logique de Hoare pour,
à partir d'un ensemble de dynamiques possibles du système (représentées par un programme impératif)
et d'un ensemble d'états finals du système (représentés par une post-condition)
obtenir la plus large condition nécessaire sur les paramètres et l'état initial, par calcul de la plus faible pré-condition,
permettant que l'ensemble des dynamiques données aboutisse dans l'ensemble des états finals donnés.
\todo{Revoir}
Cette méthode se distingue par la possibilité d'inférer, en plus d'un ensemble d'états initiaux, une classe de paramétrisations pour lesquelles les dynamiques données sont possibles.
De plus, sa complexité est fortement réduite par rapport à un calcul du Graphe des États.
Cependant, son aboutissement n'avait été que partiel face à certaines difficultés d'implémentation, bien que l'essai eut été effectué avec trois approches différentes :
\begin{itemize}
  \item une approche mathématique sous forme de preuves avec Coq,
  \item une approche fonctionnelle en OCaml,
  \item une approche logique en Prolog.
\end{itemize}

Fort d'un recul supplémentaire après un an de doctorat et de la connaissance d'ASP acquise au National Institute of Informatics, j'ai pu me replonger dans ce problème afin de proposer une approche et une implémentation efficaces.
Cette nouvelle implémentation se base sur une représentation arborescente (et non fonctionnelle) des propriétés (pré-condition et post-condition) ce qui permet, contrairement aux implémentations précédentes, de les manipuler aisément.
Cette nouvelle approche se compose de trois modules aux objectifs distincts :
\begin{itemize}
  \item Le noyau est écrit en OCaml et permet de calculer la plus faible pré-condition à partir du couple programme et post-condition donné ;
  \item L'énumération des solutions possibles (état initial et paramétrisation) peut être réalisé à l'aide d'un programme ASP représentant la pré-condition obtenue ;
  \item La fonction de simplification, ajoutée au noyaux en OCaml, permet enfin de simplifier la pré-condition obtenue lorsque certains termes sont trivialement vrais ou faux, ou l'affaiblir en fonction de contraintes sur les paramètres et la condition initiale ; cette simplification s'avère utile pour manipuler la propriété obtenue sans avoir à calculer explicitement la classe de solutions sous-jacente.
\end{itemize}

Cette nouvelle implémentation n'a pas encore fait l'objet d'une publication ou d'une communication, faute de la publication de l'article théorique initial.
Cependant, elle est fonctionnelle et les résultats sont satisfaisants car elle permet de répondre aux divers exemples proposés ainsi qu'à des cas appliqués plus complexes.
Elle permet actuellement de répondre \todo{en moins d'une seconde} là où les anciennes implémentations sont fortement limitées :
\begin{itemize}
  \item en temps lorsqu'il faut calculer le Graphe des États pour chaque paramétrisation possible,
  \item en mémoire lorsqu'il faut énumérer explicitement tous les états initiaux et paramétrisations en même temps.
\end{itemize}

\section{CS2Bio}
\label{sec:cs2bio}
\todo{Priorités fixes de mises à jour de SC -> réponse à la perspective “Réflexion sur les propriétés d'atteignabilité” + RRB}

\section{TCS}
\todo{En cours}

\chapter{Perspectives}

\todo{Revoir}

\section{Données chronométriques}
À partir de modèles algébriques tels que le Process Hitting ou le modèle de Thomas, il est possible d'extraire des propriétés à caractère chronologique,
par l'étude du Graphe des États ou à l'aide d'analyses statiques permettant de conclure quant à des atteignabilités successives de processus.
Cependant, il peut être nécessaire de contraindre ou d'étudier un système en fonction de données chronométriques, portant sur la durée des processus mis en jeu plutôt que sur leur simple succession.
L'une des perspectives de ce sujet serait ainsi d'étudier les possibilités d'étendre la sémantique du Process Hitting afin de pouvoir y inclure des données chronométriques discrètes ou continues.

%Une telle extension pourrait revêtir différents buts.
L'un des principaux buts d'une telle extension est l'inférence des délais au sein d'un modèle afin d'obtenir des résultats temporels sur certains phénomènes difficilement reproductibles ou observables expérimentalement.
L'étude du cycle circadien, qui fait partie des thématiques étudiées par MeForBio \textit{via} notamment le projet CirClock, est un exemple de domaine qui bénéficierait directement d'un tel développement.
D'autres objectifs peuvent être dégagés de cette extension, comme l'ajout d'un système d'horloges à la modélisation permettant la représentation de comportements plus complexes et dépendant du temps.

Cette perspective s'inscrit dans la collaboration en cours de développement avec l'équipe du professeur Katsumi Inoue du National Institue of Informatics,
dont le but est de développer des outils permettant l'inférence de données chronométriques à l'aide d'outils SAT.
%dans le cadre d'un projet STIC-Asie SATTIBIS

%Projet avec le Japon
%+ sujet doctorat



\section{Réflexion sur les propriétés d'atteignabilité}

L'une des forces principales du Process Hitting repose en les outils d'analyse statique permettant de vérifier des propriétés d'atteignabilité.
Cependant, ces outils s'appliquent uniquement à la sémantique de base du Process Hitting ;
les ajouts à la sémantique entraînent généralement l'impossibilité d'utiliser ces outils.
Une perspective intéressante serait ainsi de mener une réflexion approfondie sur l'application de ces outils à d'autres sémantiques.

Le Process Hitting à priorités fixes, au moins dans certaines restrictions de son utilisation, pourrait notamment bénéficier de ces outils.
Leur application aux autres sémantiques présentées à la section \ref{sec:semantiques} pourrait pourrait être intéressante sous réserve que ces sémantiques présentent un intérêt pour certaines modélisation.
Enfin, cette réflexion devra être menée durant tout le processus de développement d'une sémantique de Process Hitting intégrant des données chronométriques,
afin d'en permettre une exploitation efficace.

\todo{Priorités multiples avec précédence}

\todo{Traduction CIF2PH (Carito)}

\todo{Multiplexes avec délais (Morgan)}

\chapter{Activités scientifiques}

\newcommand{\rev}{avec \textit{peer review}}

\newcommand{\ifnotempty}[2]{\ifthenelse{\equal{#1}{}}{}{#2}}
\newcommand{\replace}[3]{\ifthenelse{\equal{#1}{#2}}{#3}{#1}}

\newcommand{\entree}[3]{\begin{samepage}\vspace{.3em}\begin{itemize}\item[$\bullet$]\textbf{#1}\quad#2\end{itemize}\vspace{.3em}%
#3\vspace{.7em}\end{samepage}}

%\newcommand{\publi}[8]{\samepage{\vspace{.3em}\begin{itemize}\item[$\bullet$]\textbf{#1}\quad#2\end{itemize}\vspace{.3em}%
%\ifnotempty{#3}{#3. }%
%\ifnotempty{#4}{#4\ifnotempty{#5#6#7}{, }}%
%\ifnotempty{#6}{\replace{#5}{in}{in : }\textit{#6}\ifnotempty{#7}{, }}%
%#7\ifnotempty{#4#5#6#7}{.}\ifnotempty{#3#4#5#6#7}{\ifnotempty{#8}{\\}}%
%#8\ifnotempty{#3#4#5#6#7#8}{\vspace{.7em}}}}

\newcommand{\publi}[8]{\entree{#1}{#2}{%
\ifnotempty{#3}{#3. }%
\ifnotempty{#4}{#4\ifnotempty{#5#6#7}{, }}%
\ifnotempty{#6}{\replace{#5}{in}{in : }\textit{#6}\ifnotempty{#7}{, }}%
#7\ifnotempty{#4#5#6#7}{.}\ifnotempty{#3#4#5#6#7}{\ifnotempty{#8}{\\}}%
#8}}

\newcommand{\expose}[3]{\publi{#1}{}{#2}{}{}{}{#3}{}}



\section{Publications}\label{sec:publications}

\subsection{Publications 2012/2013}

\publi
{Article accepté en workshop \rev}{\cite{FPMR13-CS3Bio}}
{\textbf{Maxime Folschette}, Loïc Paulevé, Morgan Magnin, Olivier Roux}
{Under-approximation of reachability in multivalued asynchronous networks}
{in}{CS2Bio’13: 4th International Workshop on Interactions between Computer Science and Biology}
{Florence, Italie, juin 2013}
{Préimpression : \url{http://www.irccyn.ec-nantes.fr/~folschet/Folschette_CS2Bio13.pdf}}



\subsection{Séminaires 2012/2013}

\publi
{Résumé étendu \rev}{}
{\textbf{Maxime Folschette}}
{Inferring Biological Regulatory Networks from Process Hitting models}
{}{MOVEP'12 : The 10th school for young researchers about Modelling and Verifying Parallel processes}
{Marseille, France, décembre 2012}
{Préimpression : \url{http://www.irccyn.ec-nantes.fr/~folschet/Folschette_MOVEP12.pdf}}

\publi
{Résumé étendu \rev}{}
{\textbf{Maxime Folschette}}
{Introduction to the Process Hitting and inference of its underlying Biological Regulatory Network}
{}{ASSB'13 : Thematic Research School on the Advances in Systems and Synthetic Biology}
{La Colle-sur-Loup, France, mars 2013}
{Préimpression : \url{http://www.irccyn.ec-nantes.fr/~folschet/Folschette_ASSB13.pdf}}

\publi
{Résumé étendu et poster}{}
{\textbf{Maxime Folschette}}
{Presentation of the Process Hitting framework and inference of Biological Regulatory Networks with Thomas parameters}
{}{JDOC : 13\textsuperscript{e} Journée des Doctorants de l'ED STIM}
{Saint-Nazaire, France, avril 2013}
{Actes et posters : \url{https://sites.google.com/site/jdoc2013fr/}}



\subsection{Publications précédentes}\label{ssec:publications}

\publi
{Article accepté en conférence \rev}{\cite{FPIMR12-CMSB}}
{\textbf{Maxime Folschette}, Loïc Paulevé, Katsumi Inoue, Morgan Magnin, Olivier Roux}
{Concretizing the Process Hitting into Biological Regulatory Networks}
{in}{CMSB'12: Proceedings of the 10th International Conference on Computational Methods in Systems Biology}
{Londres, Royaume-Uni, ACM, octobre 2012}
{\url{http://link.springer.com/chapter/10.1007\%2F978-3-642-33636-2_11}}

\publi
{Article accepté en workshop \rev}{\cite{FPIMR12-LDSSB}}
{\textbf{Maxime Folschette}, Loïc Paulevé, Katsumi Inoue, Morgan Magnin, Olivier Roux}
{Abducting Biological Regulatory Networks from Process Hitting models}
{in}{LDSSB'12: ECML/PKDD 2012 Workshop on Learning and Discovery in Symbolic Systems Biology}
{University of Bristol, Royaume-Uni, septembre 2012}
{Actes : \url{http://www.cs.bris.ac.uk/~oray/LDSSB12/LDSSB-2012.pdf}}

\publi
{Précédente publication en journal \rev}{dans le domaine de la fusion nucléaire}
{Andrea Murari, Didier Mazon, Michela Gelfusa, \textbf{Maxime Folschette}, Thibaut Quilichini et collaborateurs EFDA-JET}
{Residual analysis of the equilibrium reconstruction quality on JET}
{}{Nuclear Fusion}
{volume 51, numéro 5, avril 2011}%, DOI 10.1088/0029-5515/51/5/053012}
{\url{http://iopscience.iop.org/0029-5515/51/5/053012/}}



\section{Exposés invités}

\subsection{Exposés avec article 2012/2013}

\expose
{MOVEP'12 : The 10th school for young researchers about Modelling and Verifying Parallel processes}
{Inferring Biological Regulatory Networks from Process Hitting models}
{Marseille, France, décembre 2012}

\expose
{ASSB'13 : Thematic Research School on the Advances in Systems and Synthetic Biology}
{Introduction to the Process Hitting and inference of its underlying Biological Regulatory Network}
{La Colle-sur-Loup, France, mars 2013}

\todo{Ajouter l'exposé à Stefen Klamt ?}

\todo{Ajouter les présentations à CMSB et LDSSB ?}



\subsection{Exposés précédents}

\expose{Réunion du groupe de travail ANR BioTempo/G2 : « Des dynamiques discrètes à des dynamiques continues (modèles hybrides) »}
{Concretizing Process Hitting models into Biological Regulatory Networks with Thomas' formalism using ASP}
{Juin 2012, Nantes, France}

\expose{Séminaire informel AED}
{Modeling and Analysis of Large Biological Regulatory Networks thanks to the “Process Hitting” Framework}
{Juin 2012, Nantes, France}

\expose{Fourth CSPSAT \& ASP Seminar}
{Concretizing Process Hitting models into Biological Regulatory Networks with Thomas' formalism using ASP}
{Mai 2012, Kobe, Japon}

\expose{KUBIC-NII Joint Seminar on Bioinformatics 2012}
{Translating Process Hitting models to Thomas' modeling with ASP}
{Avril 2012, Kyoto, Japon}

\expose{The 8th Meeting on Inference-based Hypothesis-finding and its Application to Systems Biology}
{Modeling and Analysis of Large Biological Regulatory Networks thanks to the Process Hitting Framework}
{Mars 2012, Kanazawa, Japon}



\section{Collaboration}

\entree
{Katsumi Inoue}{Inoue Lab., National Institute of Informatics, Tokyo, Japon}
{Collaboration nouée lors du stage doctoral à l'Inoue Lab. en première année (mars à mai 2012).
Participation à la révision de~\cite{FPIMR12-CMSB} pour sa soumission à Theoretical Computer Science (cf.~section~\ref{sec:tcs}).}

\chapter{Autres activités}

\newcommand{\enseignement}[5]{\entree{}%$\bullet$}
{#1}{#2}{%
\ifnotempty{#3}{#3\ifnotempty{#4}{\\}}%
\ifnotempty{#4}{\textbf{Responsable :} #4}%
\begin{itemize}\renewcommand{\labelitemi}{\qquad$\bullet$}#5\end{itemize}}}

\newcommand{\formation}[3]{\entree{$\bullet$}{#1}{(#2)}{#3}}

\section{Enseignements dispensés durant l'année scolaire 2012/2013}

J'ai effectué une activité complémentaire d'enseignement (monitorat) à l'École Centrale de Nantes auprès d'élèves de 1\textsuperscript{ère}, 2\textsuperscript{e} et 3\textsuperscript{e} année (Bac +3 à Bac +5). Les heures de TP sont décomptées comme équivalentes aux heures de TD.

\bigskip

\enseignement
{Méthodes logicielles (MELOG)}{2\textsuperscript{e} année (semestre~7)}
{Programmation orientée objet, structures de données et langage Java}
{Guillaume MOREAU}
{\item 1 groupe de TD, TP et TA, soit 28 heures}

\enseignement
{Algorithmique et programmation (ALGPR)}{1\textsuperscript{ère} année (semestre~6)}
{Introduction à l'algorithmique et applications au langage C}
{Vincent TOURRE}
{\item 1 groupe de TD, TP et TA, soit 30 heures}

\enseignement
{Projet d'application (PAPPL)}{3\textsuperscript{e} année, option informatique (semestre~8)}
{Projet informatique appliqué}
{Didier LIME}
{\item 1 projet de 3 étudiants en co-encadrement, soit environ 5 heures}

\enseignement
{Projet de groupe (PGROU)}{3\textsuperscript{e} année, option informatique (semestre~8)}
{Projet informatique en groupe}
{Guillaume MOREAU}
{\item 1 projet de 4 étudiants en co-encadrement, soit environ 4 heures}

\enseignement
{Projet de recherche et développement (R\&D)}{3\textsuperscript{e} année, option R\&D (semestre~8)}
{Projet à caractère scientifique}
{Ina TARALOVA}
{\item 1 projet d'1 étudiant en co-encadrement, soit environ 2 heures}

Ma charge d'enseignement pour cette année scolaire s'élève donc à environ 69~heures, pour 64~heures requises par mon contrat.
L'année dernière, j'avais effectué environ 68~heures d'enseignement.



\section{Projets encadrés durant l'année scolaire 2012/2013}

J'ai eu l'occasion d'encadrer trois projets d'étudiants durant cette année scolaire. Tous les étudiants concernés étaient en 3\textsuperscript{e} année à l'École Centrale de Nantes.

Deux groupes d'étudiants (PGROU et PAPPL ci-avant) et se sont concentré sur l'interface graphique en développement gPH dont le but est la manipulation graphique des modèles en Process Hitting et l'interfaçage avec les outils de Pint.
Ces deux projets ont permis d'aboutir à une version stable de l'interface permettant d'effectuer des tâches de base :
\begin{itemize}
  \item proposer une représentation graphique dynamique d'un modèle de Process Hitting,
  \item associer divers attributs graphiques (couleur, position) aux sortes et les répartir en groupes,
  \item éditer la version texte du modèle et permettre une mise à jour en direct de la version graphique,
  \item définir et appeler des branchements vers des outils extérieurs (de la bibliothèque Pint notamment).
\end{itemize}
L'objectif de cette application est de permettre l'utilisation du Process Hitting par des non-informaticiens ou d'offrir une démonstration visuelle du formalisme et de ses capacités.

Le dernier projet (R\&D ci-avant), mené par un seul étudiant, a porté sur la traduction d'un modèle de Thomas avec multiplexes vers un modèle de Thomas sans multiplexe, en vue de la traduction de celui-ci en réseau de Petri.
Ce projet a permis de soulever des problématiques intéressantes liées à ce problème précis, et la traduction a été formalisée et démontrée par l'étudiant.



\section{Modules d'apprentissage de l'école doctorale}

Cette deuxième année de doctorat a été l'occasion de compléter les formations requises par l'école doctorale,
et de rattraper le retard engendré l'an passé par le stage doctoral de trois mois au Japon.
L'école doctorale STIM requiert trois formations à caractère scientifique et trois formations à caractère professionnel,
et j'ai profité de ma participation à deux écoles thématiques pour valider une partie des formations scientifiques.

\subsection{Modules scientifiques}

\formation
{MOVEP'12 : \textit{School for young researchers about Modelling and Verifying Parallel processes}}{en attente de validation}
{L'école de jeunes chercheurs MOVEP a pour but de réunir des chercheurs, des doctorants et des industriels travaillant dans les domaines du contrôle et de la vérification formelle de système parallèles. Elle se déroule sur une semaine (du 3 au 7 décembre) et se décline en tutoriels, présentations techniques et présentations de doctorants.

Le thème de cette école de jeunes chercheurs s'inscrit bien dans mon champ de recherche, qui est l'application de méthodes formelles à des modèles biologiques algébriques. J'ai de plus eu la possibilité de présenter mon sujet et mon travail passé dans le cadre des présentations de doctorants.

Plus d'informations à : \url{http://movep.lif.univ-mrs.fr/}}

\formation
{ASSB'13 : \textit{Advances in Systems and Synthetic Biology Thematic Research School}}{en attente de validation}
{L'école thématique ASSB a pour but de réunir des chercheurs dans le domaine de la biologie des systèmes et de la biologie synthétique. Elle se déroule sur une semaine (du 25 au 29 mars) et s'articule autour de conférences et d'ateliers.

Cette école thématique a été l'occasion d'entendre et de rencontrer des chercheurs du domaine dont j'avais parfois déjà lu des articles (bio-informatique et biologie des systèmes). Les exposés brassaient des thèmes assez variés, permettant notamment une ouverture vers la biologie que ma formation initiale ne m'avait pas offerte. Cela m'a aussi permis de prendre quelques contacts dans le milieu. Enfin, j'ai eu la possibilité de présenter mon sujet et mon travail passé dans le cadre des présentations de doctorants.

Plus d'informations à : \url{http://epigenomique.free.fr/en/index.php}}

\formation
{JDOC'13 : Journée des doctorants de l'école doctorale STIM}{validé}
{La Journée des Doctorants (JDOC) est une manifestation destinée à promouvoir les rencontres entre les doctorants de l'école doctorale et à échanger sur leurs travaux de thèse en cours de préparation avec l'ensemble des autres doctorants, directeurs de recherche et co-encadrants de l'école doctorale STIM. Elle concerne et implique la participation des doctorants de STIM inscrits en 2\textsuperscript{e} année ; c'est une journée d'échange et d'apprentissage de la communication vers un public de scientifiques non spécialisés.

Objectifs :
\begin{itemize}
  \item rapprocher les doctorants des différentes spécialités,
  \item faire connaître et valoriser les travaux de chacun,
  \item favoriser les échanges.
\end{itemize}

Plus d'informations à : \url{https://sites.google.com/site/jdoc2013ang/}}



\subsection{Modules professionnels}

\formation
{Communication orale : présenter son projet de thèse en temps limité}{validé}
{Les objectifs de ce module sont les suivants :
\begin{itemize}
  \item Résumer en vingt minutes son projet de thèse au moyen de « Power Point », devant un auditoire pluridisciplinaire ;
  \item Préparation de l'exposé : apprendre à sélectionner ce qui est essentiel et important ; structurer son exposé pour capter et garder l'attention de son auditoire ;
  \item Apprendre à se mettre en scène et mise en situation ;
  \item Vidéo-scopie, analyses critiques, autoévaluation.
\end{itemize}

Ce module est l'occasion d'apprendre à s'adresser à un public divers et éventuellement non spécialisé (soutenance de thèse, réunion de laboratoire, entretien d'embauche, etc.).
Il m'a aussi permis de m'entraîner à présenter mon sujet devant un public provenant de disciplines diverses (droit, lettres, sciences sociales, biologie, STAPS et informatique).}

\formation
{Découverte du journalisme scientifique avec initiation pratique au journalisme radio}{validé}
{L'objectif principal de ce module est d'initier les jeunes chercheurs de toutes les disciplines au journalisme scientifique et à la vulgarisation des sciences par la pratique, \textit{via} la réalisation d'une émission radio

Outre la découverte du métier du journaliste scientifique et la sensibilisation à l'exercice de la vulgarisation, cette initiation au journalisme scientifique s'appuie sur toutes les différentes étapes de la réalisation d'une émission scientifique, commençant par la recherche, passant par l'écriture et l'enregistrement de voix à la diffusion d'une émission radio. Ce module de formation invite chaque participant à prendre du recul par rapport à ses travaux de recherche et par rapport à sa discipline lors de l'écriture d'articles, de réalisation d'interviews et de reportages sur un sujet choisi (thématique de recherche traitée dans un laboratoire de la région).

L'émission de radio réalisée lors de ce module, intitulée \og Comprendre le monde quantique \fg, a été diffusée dans \textit{Le Labo des Savoirs} de la radio nantaise Prun'.
Le podcast de l'émission est gratuitement disponible à l'écoute sur le site Internet de la radio,
à l'adresse : \url{http://www.prun.net/emissions/le-labo-des-savoirs-07052013}}

\formation
{Anglais pour la recherche}{en cours}
{L'objectif de ce module est de tendre vers l'autonomie de l'anglais pour les sciences et acquérir des connaissances et des compétences s'appliquant au domaine professionnel :
\begin{enumerate}
  \item Prise de parole en public :
  \begin{itemize}
    \item préparation à la présentation de travaux en congrès,
    \item présentation d'une communication de 15' sur son sujet de thèse ;
  \end{itemize}
  \item Révision des notions grammaticales et lexicales de base de l'anglais scientifique avec l'ouvrage \textit{Minimum Competence in Scientific English}.
\end{enumerate}}



\section{Activités d'animation de la recherche}
\begin{itemize}
\renewcommand{\labelitemi}{\qquad$\bullet$}
  \item Membre actif de l'\emph{Association des Étudiants en Doctorat} sur le campus de l'École Centrale de Nantes (AED).
  \item Participation à la réalisation d'un cours d'initiation en trois séances à \LaTeX{} auprès des élèves en doctorat et master sur le campus de l'École Centrale de Nantes.
\end{itemize}


\bibliographystyle{plain}%alpha}
\bibliography{biblio}

%\newpage
%\todo{ANNEXES : ARCS NEUTRALISANTS + PH2THOMAS}

\end{document}
