\documentclass[11pt]{report}

\usepackage{hyperref}

\usepackage[french]{babel}
\usepackage[utf8]{inputenc}
\usepackage[T1]{fontenc}

\usepackage{amsmath}  % Maths
\usepackage{amsfonts} % Maths
\usepackage{amssymb}  % Maths
\usepackage{stmaryrd} % Maths (crochets doubles)

\usepackage{url}     % Mise en forme + liens pour URLs
\usepackage{array}   % Tableaux évolués

\usepackage{comment}

\usepackage[hmargin=3cm,vmargin=2.5cm]{geometry}

\usepackage[font=footnotesize]{caption}  % Style des légendes

\usepackage{tikz}
\newdimen\pgfex
\newdimen\pgfem
\usetikzlibrary{arrows,shapes,shadows,scopes}
\usetikzlibrary{positioning}
\usetikzlibrary{matrix}
\usetikzlibrary{decorations.text}
\usetikzlibrary{decorations.pathmorphing}

% Macros relatives à la traduction de PH avec arcs neutralisants vers PH à k-priorités fixes

% Macros générales
\def\Pint{\textsc{PINT}}

% Notations générales pour PH
\newcommand{\PH}{\mathcal{PH}}
\newcommand{\PHs}{\Sigma}
\newcommand{\PHl}{L}
%\newcommand{\PHp}{\mathcal{P}}
\newcommand{\PHp}{\textcolor{red}{\mathcal{P}}}
\newcommand{\PHproc}{\mathcal{P}}
\newcommand{\PHa}{\PHh}
\newcommand{\PHh}{\mathcal{H}}
\newcommand{\PHn}{\mathcal{N}}

\newcommand{\PHhitter}{\mathsf{hitter}}
\newcommand{\PHtarget}{\mathsf{target}}
\newcommand{\PHbounce}{\mathsf{bounce}}
\newcommand{\PHsort}{\Sigma}

\def\f#1{\mathsf{#1}}
\def\focals{\f{focals}}
\def\play{\cdot}
\def\configs#1{\mathbb C_{#1\rightarrow a}}

%\newcommand{\PHfrappeR}{\textcolor{red}{\rightarrow}}
%\newcommand{\PHmonte}{\textcolor{red}{\Rsh}}

\newcommand{\PHfrappeA}{\rightarrow}
\newcommand{\PHfrappeB}{\Rsh}
%\newcommand{\PHfrappe}[3]{\mbox{$#1\PHfrappeA#2\PHfrappeB#3$}}
%\newcommand{\PHfrappebond}[2]{\mbox{$#1\PHfrappeB#2$}}
\newcommand{\PHfrappe}[3]{#1\PHfrappeA#2\PHfrappeB#3}
\newcommand{\PHfrappebond}[2]{#1\PHfrappeB#2}
\newcommand{\PHobjectif}[2]{\mbox{$#1\PHfrappeB^*\!#2$}}
\newcommand{\PHconcat}{::}
\newcommand{\PHneutralise}{\rtimes}

\def\PHget#1#2{{#1[#2]}}
%\newcommand{\PHchange}[2]{#1\langle #2 \rangle}
\newcommand{\PHchange}[2]{(#1 \Lleftarrow #2)}
\newcommand{\PHarcn}[2]{\mbox{$#1\PHneutralise#2$}}
\newcommand{\PHjoue}{\cdot}

\newcommand{\PHetat}[1]{\mbox{$\langle #1 \rangle$}}


% Notations spécifiques à ce papier
\newcommand{\PHdirectpredec}[1]{\PHs^{-1}(#1)}
\newcommand{\PHpredec}[1]{\f{pred}(#1)}
\newcommand{\PHpredecgene}[1]{\f{reg}({#1})}
\newcommand{\PHpredeccs}[1]{\PHpredec{#1} \setminus \Gamma}

\newcommand{\PHincl}[2]{#1 :: #2}

\def\ctx{\varsigma}
\def\ctxOverride{\Cap}
\def\state#1{\langle #1 \rangle}

% Notations spécifiques aux graphes d'états
%\newcommand{\PHge}{\textcolor{red}{\mathcal{GE}}}
%\newcommand{\PHt}{\mathcal{T}}
%\newcommand{\GE}{\mathcal{GE}}
%\newcommand{\GEt}{\mathcal{T}}
%\newcommand{\GEl}{\PHl}
%\newcommand{\GEa}{\PHa}
%\newcommand{\GEva}[3]{#1 \stackrel{#2}{\longrightarrow} #3}
%\newcommand{\GEval}[3]{#1 \stackrel{#2}{\Longrightarrow} #3}
%\newcommand{\GEget}[2]{\PHget{#1}{#2}}


\def\DEF{\stackrel{\Delta}=}
\def\EQDEF{\stackrel{\Delta}\Leftrightarrow}

\def\intervalless{<_{[]}}


\usepackage{ifthen}
\usepackage{tikz}
\usetikzlibrary{arrows,shapes}

\definecolor{lightgray}{rgb}{0.8,0.8,0.8}
\definecolor{lightgrey}{rgb}{0.8,0.8,0.8}

\tikzstyle{boxed ph}=[]
\tikzstyle{sort}=[fill=lightgray,rounded corners]
\tikzstyle{process}=[circle,draw,minimum size=15pt,fill=white,
font=\footnotesize,inner sep=1pt]
\tikzstyle{black process}=[process, fill=black,text=white, font=\bfseries]
\tikzstyle{gray process}=[process, draw=black, fill=lightgray]
\tikzstyle{current process}=[process, draw=black, fill=lightgray]
\tikzstyle{process box}=[white,draw=black,rounded corners]
\tikzstyle{tick label}=[font=\footnotesize]
\tikzstyle{tick}=[black,-]%,densely dotted]
\tikzstyle{hit}=[->,>=angle 45]
\tikzstyle{selfhit}=[min distance=30pt,curve to]
\tikzstyle{bounce}=[densely dotted,->,>=latex]
\tikzstyle{hl}=[font=\bfseries,very thick]
\tikzstyle{hl2}=[hl]
\tikzstyle{nohl}=[font=\normalfont,thin]

\newcommand{\currentScope}{}
\newcommand{\currentSort}{}
\newcommand{\currentSortLabel}{}
\newcommand{\currentAlign}{}
\newcommand{\currentSize}{}

\newcounter{la}
\newcommand{\TSetSortLabel}[2]{
  \expandafter\repcommand\expandafter{\csname TUserSort@#1\endcsname}{#2}
}
\newcommand{\TSort}[4]{
  \renewcommand{\currentScope}{#1}
  \renewcommand{\currentSort}{#2}
  \renewcommand{\currentSize}{#3}
  \renewcommand{\currentAlign}{#4}
  \ifcsname TUserSort@\currentSort\endcsname
    \renewcommand{\currentSortLabel}{\csname TUserSort@\currentSort\endcsname}
  \else
    \renewcommand{\currentSortLabel}{\currentSort}
  \fi
  \begin{scope}[shift={\currentScope}]
  \ifthenelse{\equal{\currentAlign}{l}}{
    \filldraw[process box] (-0.5,-0.5) rectangle (0.5,\currentSize-0.5);
    \node[sort] at (-0.2,\currentSize-0.4) {\currentSortLabel};
   }{\ifthenelse{\equal{\currentAlign}{r}}{
     \filldraw[process box] (-0.5,-0.5) rectangle (0.5,\currentSize-0.5);
     \node[sort] at (0.2,\currentSize-0.4) {\currentSortLabel};
   }{
    \filldraw[process box] (-0.5,-0.5) rectangle (\currentSize-0.5,0.5);
    \ifthenelse{\equal{\currentAlign}{t}}{
      \node[sort,anchor=east] at (-0.3,0.2) {\currentSortLabel};
    }{
      \node[sort] at (-0.6,-0.2) {\currentSortLabel};
    }
   }}
  \setcounter{la}{\currentSize}
  \addtocounter{la}{-1}
  \foreach \i in {0,...,\value{la}} {
    \TProc{\i}
  }
  \end{scope}
}

\newcommand{\TTickProc}[2]{ % pos, label
  \ifthenelse{\equal{\currentAlign}{l}}{
    \draw[tick] (-0.6,#1) -- (-0.4,#1);
    \node[tick label, anchor=east] at (-0.55,#1) {#2};
   }{\ifthenelse{\equal{\currentAlign}{r}}{
    \draw[tick] (0.6,#1) -- (0.4,#1);
    \node[tick label, anchor=west] at (0.55,#1) {#2};
   }{
    \ifthenelse{\equal{\currentAlign}{t}}{
      \draw[tick] (#1,0.6) -- (#1,0.4);
      \node[tick label, anchor=south] at (#1,0.55) {#2};
    }{
      \draw[tick] (#1,-0.6) -- (#1,-0.4);
      \node[tick label, anchor=north] at (#1,-0.55) {#2};
    }
   }}
}
\newcommand{\TSetTick}[3]{
  \expandafter\repcommand\expandafter{\csname TUserTick@#1_#2\endcsname}{#3}
}

\newcommand{\myProc}[3]{
  \ifcsname TUserTick@\currentSort_#1\endcsname
    \TTickProc{#1}{\csname TUserTick@\currentSort_#1\endcsname}
  \else
    \TTickProc{#1}{#1}
  \fi
  \ifthenelse{\equal{\currentAlign}{l}\or\equal{\currentAlign}{r}}{
    \node[#2] (\currentSort_#1) at (0,#1) {#3};
  }{
    \node[#2] (\currentSort_#1) at (#1,0) {#3};
  }
}
\newcommand{\TSetProcStyle}[2]{
  \expandafter\repcommand\expandafter{\csname TUserProcStyle@#1\endcsname}{#2}
}
\newcommand{\TProc}[1]{
  \ifcsname TUserProcStyle@\currentSort_#1\endcsname
    \myProc{#1}{\csname TUserProcStyle@\currentSort_#1\endcsname}{}
  \else
    \myProc{#1}{process}{}
  \fi
}

\newcommand{\repcommand}[2]{
  \providecommand{#1}{#2}
  \renewcommand{#1}{#2}
}
\newcommand{\THit}[5]{
  \path[hit] (#1) edge[#2] (#3#4);
  \expandafter\repcommand\expandafter{\csname TBounce@#3@#5\endcsname}{#4}
}
\newcommand{\TBounce}[4]{
  (#1\csname TBounce@#1@#3\endcsname) edge[#2] (#3#4)
}

\newcommand{\TState}[1]{
  \foreach \proc in {#1} {
    \node[current process] (\proc) at (\proc.center) {};
  }
}

% Macros spécifiques au Modèle de Thomas / aux RRB

% Notations pour le modèle de Thomas (depuis thèse)
\newcommand{\GRN}{\mathcal{GRN}}
\newcommand{\IG}{\mathcal{G}}
%\def\IG{\mathrm{IG}}
\newcommand{\GRNreg}[1]{\Gamma^{-1}(#1)}
\newcommand{\GRNres}[2]{\mathsf{Res}_{#1}(#2)}
\newcommand{\GRNallres}[1]{\mathsf{Res}_{#1}}
\newcommand{\GRNget}[2]{\PHget{#1}{#2}}
\newcommand{\GRNetat}[1]{\PHetat{#1}}

\def\levels{\mathsf{levels}}
\def\levelsA#1#2{\levels_+(#1\rightarrow #2)}
\def\levelsI#1#2{\levels_-(#1\rightarrow #2)}
%\newcommand{\PHres}{\mathsf{Res}}

\newcommand{\Kinconnu}{\emptyset}
\newcommand{\RRGva}[3]{#1 \stackrel{#2}{\longrightarrow} #3}
\newcommand{\RRGgi}{\mathcal{G}}
\newcommand{\RRGreg}[1]{\RRGgi_{#1}}



%\definecolor{darkred}{rgb}{0.5,0,0}
%\definecolor{lightred}{rgb}{1,0.8,0.8}
%\definecolor{lightgreen}{rgb}{0.7,1,0.7}
\definecolor{darkgreen}{rgb}{0,0.5,0}
%\definecolor{darkyellow}{rgb}{0.5,0.5,0}
%\definecolor{lightyellow}{rgb}{1,1,0.6}
%\definecolor{darkcyan}{rgb}{0,0.6,0.6}
%\definecolor{darkorange}{rgb}{0.8,0.2,0}

%\definecolor{notsodarkgreen}{rgb}{0,0.7,0}

%\definecolor{coloract}{rgb}{0,1,0}
%\definecolor{colorinh}{rgb}{1,0,0}
\colorlet{coloract}{darkgreen}
\colorlet{colorinh}{red}
%\colorlet{coloractgray}{lightgreen}
%\colorlet{colorinhgray}{lightred}
%\colorlet{colorinf}{darkgray}
%\colorlet{coloractgray}{lightgreen}
%\colorlet{colorinhgray}{lightred}

%\colorlet{colorgray}{lightgray}


\tikzstyle{grn}=[every node/.style={circle,draw=black,outer sep=2pt,minimum
                size=15pt,text=black}, node distance=1.5cm]
\tikzstyle{act}=[->,draw=black,thick,color=black]
\tikzstyle{inh}=[>=|,-|,draw=black,thick, text=black,label]
%\tikzstyle{inh}=[>=|,-|,draw=colorinh,thick, text=black,label]
%\tikzstyle{act}=[->,>=triangle 60,draw=coloract,thick,color=coloract]
%\tikzstyle{inhgray}=[>=|,-|,draw=colorinhgray,thick, text=black,label]
%\tikzstyle{actgray}=[->,>=triangle 60,draw=coloractgray,thick,color=coloractgray]
\tikzstyle{inf}=[->,draw=colorinf,thick,color=colorinf]
%\tikzstyle{elabel}=[fill=none, above=-1pt, sloped,text=black, minimum size=10pt, outer sep=0, font=\scriptsize,draw=none]
\tikzstyle{elabel}=[fill=none,text=black, above=-2pt,%sloped,
minimum size=10pt, outer sep=0, font=\scriptsize, draw=none]
%\tikzstyle{elabel}=[]
\tikzstyle{sg}=[every node/.style={outer sep=2pt,minimum
                size=15pt,text=black}, node distance=2cm]



% Commandes À FAIRE
%\usepackage{color} % Couleurs du texte
%\definecolor{darkgreen}{rgb}{0,0.5,0}
%\newcommand{\afaire}[1]{\textcolor{red}{[À FAIRE~: #1]}}
%\newcommand{\resume}[1]{\textcolor{blue}{#1}}
%\newcommand{\todo}[1]{\textcolor{red}{\textbf{[TODO : #1]}}}


% Id est
\newcommand{\ie}{\textit{i.e.}~}

% Maths en gras et italique
\DeclareMathAlphabet{\mathitbf}{OML}{cmm}{b}{it}

% Symbole « non-signé »
\newcommand{\uns}{\pm}

% Mise en valeur
\newcommand{\bemph}[1]{\textbf{#1}}






\title{Rapport pour le comité de suivi de thèse 2012/2013}

\author{Maxime Folschette}

%LUNAM Universit\'e, \'Ecole Centrale de Nantes, IRCCyN UMR CNRS 6597\\
%(Institut de Recherche en Communications et Cybern\'etique de Nantes)\\
%1 rue de la No\"e -- B.P. 92101 -- 44321 Nantes Cedex 3, France.\\
%\email{Maxime.Folschette@irccyn.ec-nantes.fr}

\begin{document}

%\maketitle


\thispagestyle{empty}

~
%\vspace{5cm}
\vfill
{\Huge
\begin{center}
\textbf{Rapport pour le comité de suivi\\de thèse 2012/2013}

\vspace{1cm}

\LARGE
Modélisation algébrique de l'évolution et de la dynamique multi-échelles des réseaux de régulation biologique

\vspace{2cm}

\normalsize
20 juin 2013

\vspace{1cm}

\Large
Maxime Folschette

\vspace{2cm}

\normalsize
\begin{tabular}{rl}
  Directeur :&Olivier Roux\\
  Co-encadrant :&Morgan Magnin\\
  &\\
  &\\
  Établissement :&École Centrale de Nantes\\
  Laboratoire :&IRCCyN, UMR CNRS 6597\\
  Équipe :&MeForBio\\
  École doctorale :&STIM
\end{tabular}

\end{center}

\vspace{2cm}
}
%\vfill

\setcounter{page}{0}
\tableofcontents


\chapter{Problématique}

Chercher à modéliser les phénomènes biologiques est une étape importante dans la compréhension des mécaniques du vivant.
Au sein de la machine cellulaire, notamment, les cascades de régulations entre gènes forment des systèmes complexes que l'on cherche à comprendre et à maîtriser.
Les applications en sont nombreuses, et peuvent aller jusqu'à l'élaboration de thérapies géniques.
%L'étude et la compréhension des mécanismes du vivant, notamment au sein de la machine cellulaire, posent de nombreux problèmes de représentation des phénomènes biologiques.

Afin de permettre leur représentation et leur étude \textit{in silico}, de tels systèmes peuvent être représentés sous la forme de modèles.
%L'étude de tels systèmes de régulations nécessite l'élaboration de modèles.
Plusieurs critères participent à la qualité d'un modèle biologique, et influent sur sa facilité d'élaboration, de lecture et d'exploitation.
Le formalisme utilisé doit en effet permettre de représenter de façon pratique et complète un processus biologique donné, et donc
de proposer une sémantique claire, sans lacune et compréhensible par une large communauté.
De plus, son exploitation doit dans l'idéal être efficace, c'est-à-dire permettre l'obtention de résultats en évitant les problèmes de complexité, afin d'offrir un temps de calcul et une taille en mémoire raisonnables.

Dans ce cadre, les Réseaux de Régulation Biologiques permettent de représenter des systèmes biologiques habituellement déterminés par les physiciens en termes d'équations différentielles.
Ces équations étant souvent difficiles à résoudre, elles sont simplifiées sous la forme de systèmes algébriques dont les influences entre composants se résument à des activations et des inhibitions.
Au cours de ma thèse, je m'attache à l'étude de deux modèles algébriques en particulier : le modèle de Thomas et le Process Hitting.

\section{Modèle de Thomas}
Le modèle de Thomas, introduit en 1973 dans \cite{Thomas73}, propose une simplification cohérente des modèles continus sous forme d'équations différentielles.
Plutôt que de considérer les valeurs réelles de concentration des protéines synthétisées, ce modèle repose sur l'utilisation de seuils qui représentent des valeurs particulières de ces concentrations au-delà desquelles les influences entre composants évoluent.
En plus de proposer une approche qualitative cohérente, cette représentation permet de s'affranchir de la connaissance des valeurs réelles des concentrations, qui sont souvent difficiles à obtenir expérimentalement.

Il est possible de représenter simplement un système dans ce formalisme sous la forme d'un Graphe des Interactions dont les nœuds représentent les composants et les arcs orientés et étiquetés représentent les interactions entre ces composants.
Ce modèle introduit également la notion de paramètre discret, qui permet de spécifier la dynamique du système, notamment dans les cas de coopérations entre composants.
Un tel paramètre joue le rôle de \og point focal \fg{} dans le sens où il détermine la direction d'évolution d'un composant du système pour une configuration donnée.
La figure \ref{fig:exRRB} donne un exemple de modèle de Thomas simple.

\begin{figure}[ht]
\begin{minipage}{0.4\linewidth}
\centering
\scalebox{1.2}{
\begin{tikzpicture}[grn]
\path[use as bounding box] (0,-0.7) rectangle (3.5,0.7);
\node[inner sep=0] (a) at (2,0) {a};
\node[inner sep=0] (b) at (0,0) {b};
\node[inner sep=0] (c) at (3.5,0) {c};
%\path
%  node[elabel, below=-1em of a] {$0..2$}
%  node[elabel, below=-1em of b] {$0..1$}
%  node[elabel, below=-1em of c] {$0..1$};
\path[->]
  (b) edge[bend right] node[elabel, below=-3pt] {$+1$} (a)
  (c) edge node[elabel, above=-5pt] {$+1$} (a)
  (a) edge[bend right] node[elabel, above=-5pt] {$-2$} (b);
\end{tikzpicture}
}
\end{minipage}
\begin{minipage}{0.6\linewidth}
\centering
\begin{align*}
K_{a,\{b,c\}} &= 2 & K_{b,\emptyset} &= 1 \\
K_{a,\{b\}} &= 1 & K_{b,\{a\}} &= 0 \\
K_{a,\{c\}} &= 1 &&\\
K_{a,\emptyset} &= 0 & K_{c,\emptyset} &= 1
\end{align*}
\end{minipage}
\caption{\label{fig:exRRB}
Exemple de Réseau de Régulation Biologique, comprenant un Graphe des Interactions (à gauche) et une paramétrisation (à droite).
Chaque nœud du graphe représente un composant et chaque arc une régulation
dont l'étiquette indique le type (“$+$” pour une activation et “$-$” pour une inhibition) et le seuil.
Les paramètres tiennent lieu de points focaux pour l'état concerné :
“$K_{b,\{a\}} = 0$” signifie par exemple que $b$ évolue vers $0$ dans toutes les configurations où le niveau de $a$ est supérieur à la valeur du seuil figurant sur l'arc $a \rightarrow b$, soit $2$.
}
\end{figure}

Aujourd'hui plus largement utilisé sous sa forme multivaluée \cite{richard-comet-bernot-08}, le modèle de Thomas a connu un certain nombre d'extensions,
comme l'ajout de multiplexes \cite{bernot-comet-khalis-08},
l'intégration du temps continu \cite{Ahmad08},
ou encore l'étude de sémantiques plus expressives \cite{BernotSemBRN}.
Il a aussi été l'objet de travaux concernant la prédiction de son comportement d'après le Graphe des Interactions seul \cite{RiCo07}.

Bien qu'il soit adapté à la représentation des Réseaux de Régulation Biologiques, et que son utilisation soit très répandue, le modèle de Thomas souffre de deux principaux inconvénients.
Tout d'abord, son utilisation pour la recherche de propriétés intéressantes sur les systèmes modélisés nécessite souvent l'analyse du Graphe des États dont la construction s'avère être de complexité exponentielle.
De plus, l'étude d'un système représenté par ce formalisme nécessite une parfaite connaissance des coopérations entre composants, et donc d'avoir choisi une paramétrisation complète au sein d'un ensemble de possibilités potentiellement très important.



\hyphenation{MeForBio}

\section{Process Hitting}\label{sec:PH}
Une modélisation des Réseaux de Régulation Biologiques à l'aide de $\pi$-calcul, appelée Process Hitting (ou Frappes de Processus), a été récemment introduite par l'équipe MeForBio~\cite{PMR10-TCSB,PaulevePhD}.
Elle propose un point de vue plus modulable des influences entre composants grâce à une représentation d'actions atomiques entre ceux-ci.
Cette représentation particulière offre des possibilités d'analyse statique efficaces permettant de sur-approximer et sous-approximer l'atteignabilité d'un processus \cite{PMR12-MSCS}.
De plus, son atomicité permet d'adopter différents niveaux d'abstraction dans la modélisation, afin notamment de représenter une sur-approximation du comportement d'un système dont la spécification des coopérations ne serait pas entièrement déterminée.
Une méthode efficace de détermination des points fixes a aussi été développée.
La figure \ref{fig:exPH} donne un exemple de modèle en Process Hitting.

\tikzstyle{prio}=[draw,thick,-stealth]

\begin{figure}[ht]
  \centering
  \scalebox{1.4}{
  \begin{tikzpicture}
    \TSort{(0,0)}{a}{2}{l}
    \TSort{(0,4)}{b}{2}{l}
    \TSort{(6,2)}{c}{2}{r}

    \TSetTick{ab}{0}{00}
    \TSetTick{ab}{1}{01}
    \TSetTick{ab}{2}{10}
    \TSetTick{ab}{3}{11}
    \TSort{(3,1)}{ab}{4}{r}

    \THit{a_0}{prio}{ab_3}{.west}{ab_1}
    \THit{a_0}{prio}{ab_2}{.west}{ab_0}
    \THit{a_1}{prio}{ab_1}{.west}{ab_3}
    \THit{a_1}{prio}{ab_0}{.west}{ab_2}

    \THit{b_0}{prio}{ab_3}{.west}{ab_2}
    \THit{b_0}{prio}{ab_1}{.west}{ab_0}
    \THit{b_1}{prio}{ab_2}{.west}{ab_3}
    \THit{b_1}{prio}{ab_0}{.west}{ab_1}
    
    \THit{a_1}{selfhit}{a_1}{.west}{a_0}
    \THit{b_1}{selfhit}{b_1}{.west}{b_0}
    \THit{a_0}{bend left}{b_0}{.south west}{b_1}
    \THit{b_0}{bend right=60}{a_0}{.west}{a_1}

    \THit{ab_3}{}{c_0}{.west}{c_1}

    \path[bounce, bend right]
      \TBounce{ab_3}{}{ab_1}{.north west}
      \TBounce{ab_2}{}{ab_0}{.north west}
      \TBounce{ab_3}{}{ab_2}{.north west}
      \TBounce{ab_1}{}{ab_0}{.north west}
    ;
    \path[bounce, bend left]
      \TBounce{ab_1}{}{ab_3}{.south west}
      \TBounce{ab_0}{}{ab_2}{.south west}
      \TBounce{ab_2}{}{ab_3}{.south west}
      \TBounce{ab_0}{}{ab_1}{.south west}
    ;
    \path[bounce, bend right]
      \TBounce{a_1}{}{a_0}{.north west}
      \TBounce{b_1}{}{b_0}{.north west}
    ;
    \path[bounce, bend left]
      \TBounce{a_0}{}{a_1}{.south west}
      \TBounce{b_0}{}{b_1}{.south west}
    ;
    \path[bounce, bend left]
      \TBounce{c_0}{}{c_1}{.south west}
    ;
    \TState{a_1, b_0, ab_2, c_0}
  \end{tikzpicture}
  }
  \caption{\label{fig:exPH}
    Exemple de Process Hitting comprenant quatre sortes : $a$, $b$, $ab$ et $c$, représentées par des rectangles arrondis, contenant des cercles représentant les processus.
    Les actions sont représentées par des flèches en trait plein (frappe) puis pointillé (bond).
    La sorte $ab$ est appelée sorte coopérative (cf.~section~\ref{sec:cs2bio}).
    Les actions qui frappent cette sorte, représentées en trait épais, peuvent être priorisées comme expliqué à la section~\ref{sec:cs2bio}.
    Les processus grisés représentent un état initial possible du modèle : $\PHetat{a_1, b_0, ab_{10}, c_0}$.
  }
\end{figure}

Plusieurs extensions ont aussi été proposées pour enrichir ce formalisme.
Une première repose sur l'introduction de stochasticité afin de modéliser la durée d'évolution relative des composants à l'aide probabilités.
Cette extension nécessite l'exécution du modèle afin d'en extraite des propriétés empiriques, ou l'utilisation d'un model checker.
Une seconde extension consiste en l'attribution de classes de priorités aux actions, afin d'imposer formellement un ordre de tir entre celles-ci.
Cette extension avait été évoquée par le passé, et a fait l'objet d'un travail approfondi qui a débouché sur une publication (cf.~section~\ref{sec:cs2bio}).
%Cette sémantique reposant sur des priorités fixes permet de modéliser des comportements plus fins, par exemple au niveau des coopérations.
%Elle ne modifie pas les résultats concernant la recherche de points fixes, mais n'est pas compatible avec la méthode de sous-approximation de l'analyse statique.



\section{Enjeux}

Le développement et l'enrichissement de nouveaux formalismes possède comme but premier une meilleure représentation des systèmes et des phénomènes modélisés.
L'un des objectifs de l'équipe MeForBio est l'étude formelle des propriétés de systèmes de grande taille,
en s'appuyant notamment sur des représentations complémentaires et adaptées de ceux-ci.

Le Process Hitting répond à cette problématique en offrant une modélisation exploitable pour la représentation de modèles biologiques.
La forme particulière des actions de ce formalisme a notamment permis le développement d'outils d'analyse statique très efficaces pour la recherche de points fixes ou pour des questions d'atteignabilité.
%Base sur un système de sortes et d'actions restreintes, représentant les composants et leurs interactions,
%et en proposant des outils d'analyse statique très efficaces pour la recherche de points fixes ou pour des questions d'atteignabilité.
Ces méthodes donnent effectivement des résultats en quelques centièmes de secondes sur des modèles de grande et très grande taille (de l'ordre de 20 à 100 composants), là où les model-checkers habituels ne permettent pas toujours de répondre faute de temps ou de mémoire, ou répondent en un temps supérieur de plusieurs ordres de grandeur \cite{PMR12-MSCS}.

Ma thèse de doctorat s'inscrit dans cette logique d'étude et d'analyse des Réseaux de Régulation Biologiques de grande taille.
Le développement du Process Hitting par l'élaboration de nouveaux outils ou l'enrichissement de sa sémantique semble ainsi une voie toute tracée pour l'enrichissement de la palette d'outils répondant à ces problématiques.
Nous verrons ainsi qu'il est possible, à l'aide d'une extension cohérente de la sémantique du Process Hitting, d'étendre l'analyse statique développée dans de précédents travaux à l'ensemble des modèles de type Réseaux Discrets (cf.~section~\ref{sec:cs2bio}).
De plus, l'élaboration d'outils permettant l'extraction d'un modèle de Thomas à partir d'un modèle de Process Hitting montre le lien entre ces deux formalismes (cf.~section~\ref{sec:tcs}).

Cependant, la forme actuelle du Process Hitting ne doit pas rester la seule piste de développement possible.
Une approche par logique de Hoare démontre la possibilité d'inférer des paramètres discrets du modèle de Thomas à partir de propriétés dynamiques d'un système (cf.~section~\ref{sec:hoare}).
De plus, à l'avenir, une approche par intégration de données chronométriques pourrait notamment être envisagée (cf.~chapitre~\ref{chap:perspectives}).

\chapter{Contributions}

\newcommand{\hoare}[3]{\{\ #1\ \}\ #2\ \{\ #3\ \}}

\section{Inférence de paramètres par la logique de Hoare \normalsize(en cours)}
\label{sec:hoare}

Réalisée au sein de l'équipe MeForbio avant le début de mon doctorat, ma thèse de master avait pour sujet l'« Application de la logique de Hoare aux Réseaux de Régulation Génétiques avec multiplexes »~\cite{Folschette2011}.
Son objectif était d'étudier des travaux en cours permettant de déduire les propriétés nécessaires d'un système pour observer certains comportements dynamiques, et d'implémenter une telle analyse.
La logique de Hoare, introduite dans \cite{hoare-69}, se base des \emph{triplets de Hoare}, notés $\hoare{P}{Q}{R}$, et où :
\begin{itemize}
  \item $P$ est une pré-condition (état initial),
  \item $Q$ est un programme impératif (chemin d'exécution),
  \item $R$ est une post-condition (état final).
\end{itemize}
Un triplet $\hoare{P}{Q}{R}$ signifie que, partant d'un état initial respectant la pré-condition~$P$, l'exécution du programme~$Q$ amènera toujours le système dans un état final respectant la post-condition~$R$.
D'après les travaux de Dijkstra~\cite{dijkstra-75}, il est possible, en ne connaissant que le programme et la post-condition,
d'inférer la plus faible pré-condition possible permettant d'obtenir un tel triplet.

Les travaux sur lesquels se basent ma thèse de master utilisent ainsi la logique de Hoare pour,
à partir d'un ensemble de dynamiques possibles d'un système biologique (représentées par un programme impératif~$Q$)
et d'un ensemble d'états finals du système (représentés par une post-condition~$R$),
obtenir la plus large condition nécessaire (par calcul de la plus faible pré-condition~$P$)
telle que le système vérifie $\hoare{P}{Q}{R}$.
Cette méthode se distingue par la possibilité d'inférer, en plus d'un ensemble d'états initiaux, une classe de paramétrisations pour lesquelles les dynamiques données sont possibles.
De plus, sa complexité est fortement réduite comparée à un calcul du Graphe des États.
Cependant, son aboutissement lors du travail de master n'avait été que partiel face à certaines difficultés d'implémentation, bien que l'essai eut été effectué avec trois approches différentes :
\begin{itemize}
  \item une approche mathématique sous forme de preuves avec Coq,
  \item une approche fonctionnelle en OCaml,
  \item une approche logique en Prolog.
\end{itemize}

Fort d'un recul supplémentaire après un an de doctorat et de la connaissance d'ASP acquise au National Institute of Informatics (cf.~section~\ref{sec:collaboration}), j'ai pu me replonger dans ce problème afin de proposer une approche et une implémentation efficaces.
Cette nouvelle implémentation se base sur une représentation arborescente (et non fonctionnelle) des propriétés (pré-condition et post-condition) ce qui permet, contrairement aux implémentations précédentes, de les manipuler aisément.
Cette approche se compose de trois modules aux objectifs distincts :
\begin{itemize}
  \item Le noyau est écrit en OCaml et permet de calculer la plus faible pré-condition~$P$ à partir d'un couple~$(Q ; R)$ formé d'un programme et d'une post-condition donnés ;
  \item L'énumération des solutions possibles (sous la forme de couples~$(s ; K)$ représentant état initial et paramétrisation) peut être réalisée à l'aide d'une traduction de la pré-condition~$P$ obtenue en un programme ASP ;
  \item La fonction de simplification, ajoutée au noyaux en OCaml, permet enfin de simplifier ou d'affaiblir la pré-condition~$P$ obtenue lorsque certaines données sont connues (niveaux d'expression ou paramètres) ; cette simplification s'avère utile pour manipuler la propriété obtenue sans avoir à calculer explicitement la classe de solutions sous-jacente.
\end{itemize}

Cette nouvelle implémentation n'a pas encore fait l'objet d'une publication ou d'une communication, faute de la publication de l'article théorique initial.
Cependant, elle est fonctionnelle et semble prometteuse car elle permet de répondre aux divers exemples proposés ainsi qu'à des cas appliqués plus complexes.
Elle permet actuellement de répondre en moins d'une seconde \todo{chiffres précis} là où les précédentes implémentations sont fortement limitées :
\begin{itemize}
  \item en temps lorsqu'il faut calculer le Graphe des États pour chaque paramétrisation possible,
  \item en mémoire lorsqu'il est nécessaire d'énumérer explicitement et conjointement tous les états initiaux et toutes les paramétrisations.
\end{itemize}
En ce sens, ce travail s'intègre parfaitement à la problématique d'exploitation des grands Réseaux de Régulation Biologiques.



\section{Analyse d'atteignabilité dans un réseau discret \normalsize(publié)}
\label{sec:cs2bio}

Parmi les perspectives de l'année précédente figurait une « réflexion sur les propriétés d'atteignabilité », notamment par l'enrichissement de la sémantique du Process Hitting et l'adaptation de l'analyse statique.
En effet, l'ajout d'expressivité à la sémantique de base du Process Hitting, et notamment de priorités fixes entre les actions, rend caduque certaines propriétés d'atteignabilité, car la dynamique évolue en conséquence.

L'objectif d'une extension de sémantique est la création de modèles plus justes et plus proches de modélisations existantes.
Ainsi, la sémantique de base du Process Hitting permet une coopération entre composants à l'aide de sortes particulières appelées sortes coopératives.
L'objectif d'une sorte coopérative est de représenter chacun des états entremêlés de plusieurs composants.
Par exemple, la sorte $ab$ de la figure~\ref{fig:exPH} est une sorte coopérative sur les sortes $a$ et $b$ : chacun de ses processus représente un sous-état des états indépendants de $a$ et $b$ ($ab_{01}$ modélise la présence de $a_0$ et $b_1$, etc.).
Les actions frappant la sorte $ab$ permettent de \emph{mettre à jour} l'état de cette sorte en fonction de l'état des composants qu'elle représente ;
par exemple, les actions $\PHfrappe{a_0}{ab_{11}}{ab_{01}}$ et $\PHfrappe{a_0}{ab_{10}}{ab_{00}}$ permettent sa mise à jour lorsque le processus $a_0$ est actif.
Ainsi, l'accessibilité d'un processus de la sorte coopérative est conditionnée par l'accessibilité de chaque processus qu'il représente.
Il est alors possible d'ajouter une action déclenchée par un processus de la sorte coopérative plutôt que plusieurs actions indépendantes partant des composants qui coopèrent, comme l'action $\PHfrappe{ab_{11}}{c_0}{c_1}$ de la figure~\ref{fig:exPH}.

Un tel mécanisme à base de sortes coopératives souffre cependant d'un problème de “décalage temporel” au niveau des mises à jour.
En effet, rien de force à jouer prioritairement les actions de mise à jour des sortes coopératives par rapport aux autres actions.
Par conséquent, une sorte coopérative représente toujours l'état présent ou un état passé du système, ce autorise parfois des comportements non souhaitables.
Par exemple, dans le Process Hitting de la figure~\ref{fig:exPH}, la dynamique suivante est possible (les processus qui évoluent sont indiqués en gras) :
$$
  \PHetat{a_1, b_0, ab_{10}, c_0} \rightarrow
  \PHetat{\mathitbf{a_0}, b_0, ab_{10}, c_0} \rightarrow
  \PHetat{a_0, \mathitbf{b_1}, ab_{10}, c_0} \rightarrow
  \PHetat{a_0, b_1, \mathitbf{ab_{11}}, c_0} \rightarrow
  \PHetat{a_0, b_1, ab_{11}, \mathitbf{c_1}}
$$
En d'autres termes, le processus $ab_{11}$ peut être atteint alors que les processus $a_1$ et $b_1$ n'ont pas été présents simultanément (\ie au sein du même état).
On peut d'ailleurs montrer que si l'état initial ne contient pas $a_1$ et $b_1$, alors ceux-ci ne seront jamais présents simultanément.
Un tel comportement est possible car la mise à jour de la sorte coopérative $ab$ n'est pas forcée :
$ab_{10}$ peut rester actif même lorsque $a_1$ n'est plus actif.

Afin de pallier cela, la notion de priorités fixes avait déjà été évoquée.
Cette idée a été développée et a donné lieu à une \bemph{publication retenue à la conférence internationale CS2Bio 2013}~\cite{FPMR13-CS3Bio}.
Les apports techniques de ce travail se divisent en deux point principaux.
\begin{itemize}
  \item D'une part, une extension formelle du Process Hitting est proposée, sous la forme d'une sémantique avec mise à jour prioritaire des sortes coopératives.
  Ces priorités permettent de résoudre les problèmes de décalage temporel vus précédemment.
  Elles sont restreintes dans ce travail à l'ajout d'une seule classe d'actions priorisées, à laquelle appartiennent toutes les actions de mise à jour des sortes coopératives et qui doivent ainsi être jouées de façon prioritaire, avant toute autre action non priorisée,
  Cette classe d'actions priorisées est représentée dans l'exemple de la figure~\ref{fig:exPH} par l'ensemble des actions en trait épais :
  l'accessibilité de $ab_{11}$ est rendue impossible car lorsque le processus $a_0$ est atteint, l'action prioritaire $\PHfrappe{a_0}{ab_{10}}{ab_{00}}$ doit nécessairement être jouée avant l'action $\PHfrappe{a_0}{b_0}{b_1}$.
  \item D'autre part, il a été nécessaire d'adapter une partie des méthodes initiales d'analyse statique qui ne s'appliquent pas à cette sémantique étendue.
  La méthode de sous-approximation a été revue et prend désormais en compte le caractère priorisé des actions de mise à jour.
  La méthode de sur-approximation reste en revanche valable sur le Process Hitting “aplati” (\ie sans considérer les classes de priorité).
\end{itemize}
Il a enfin été montré que cette sémantique étendue est bisimilaire aux réseaux discrets (dont fait partie la modélisation de Thomas).
Ainsi, nous avons développé une méthode d'analyse d'atteignabilité très efficace sur les réseaux discrets booléens et multivalués.
Ces travaux ont été intégrés à la bibliothèque Pint et testés notamment sur un modèle booléen de 94 composants :
les résultats sont tous conclusifs et ont été obtenus en quelques centièmes de secondes sur un ordinateur personnel,
soit une performance bien supérieure à celle d'un model-checker standard.

Ces résultats font écho à la première étude réalisée au début de la thèse, qui avait abouti à la formalisation de deux nouvelles sémantiques plus expressives du Process Hitting.
La première introduisait la notion d'\emph{arc neutralisant} modélisant une priorité locale entre deux actions ;
il a été montré que cette sémantique est faiblement bisimilaire à la sémantique du Process Hitting avec priorités fixes.
La seconde consistait en la généralisation de la notion d'action à celle d'\emph{action conjointe} possédant plusieurs frappeurs ;
cette sémantique est faiblement bisimilaire à la sémantique présentée dans cette section car une action conjointe possède le même comportement qu'une sorte coopérative avec actions priorisées bien choisie.



\section{Extraction de Réseaux de Régulation Biologiques complets depuis un modèle de Process Hitting \normalsize(en cours)}
\label{sec:tcs}

Cette première année de thèse avait été l'occasion de compléter les liens formels entre modèle de Thomas et Process Hitting,
en publiant notamment une méthode de traduction d'un modèle en Process Hitting vers un (ensemble de) modèle(s) de Thomas dont le comportement est strictement inclus.
Ce travail avait été réalisé l'année précédente lors d'un stage doctoral de trois mois, du 1\textsuperscript{er} mars au 25 mai 2012, au National Institute of Informatics à Tokyo, supervisé par le professeur Katsumi Inoue.
Il avait abouti à une implémentation sous la forme de l'outil \texttt{ph2thomas} intégré à la bibliothèque existante Pint\footnote{Disponible à \url{http://process.hitting.free.fr}} et utilisant notamment un paradigme de programmation logique appelé \emph{Answer Set Programming} (ASP).
Cela a permis de produire une publication en workshop~\cite{FPIMR12-LDSSB} présentée en septembre 2012, une publication en conférence internationale~\cite{FPIMR12-CMSB} présentée en octobre 2012, et donne lieu aujourd'hui à une collaboration durable entre l'équipe MeForBio et l'Inoue Laboratory.
Ce travail fait actuellement l'objet d'une \bemph{révision en vue de sa soumission dans la revue Theoretical Computer Science}.

L'inférence d'un modèle de Thomas depuis un Process Hitting donné se décompose en deux étapes principales :
\begin{itemize}
  \item L'inférence du Graphe des Interactions (GI) permet d'obtenir la structure du modèle et des interactions entre composants.
  Elle doit être réalisée en premier car la paramétrisation du modèle dépend du GI.
  \item L'inférence des paramètres permet d'obtenir une paramétrisation (possiblement partielle) telle que le modèle final respecte la dynamique du Process Hitting initial (aucun comportement supplémentaire n'est possible).
\end{itemize}
Ces deux étapes peuvent échouer car certains comportement du Process Hitting ne peuvent être représentés pas un modèle de Thomas.
Ainsi, si certains paramètres n'ont pas pu être inférés lors de la seconde étape, cela signifie généralement que la dynamique du Process Hitting est trop générale pour être représentée par un unique modèle de Thomas.
Dans ce cas, une énumération des paramétrisations complètes et compatibles peut être effectuée pour rattraper cet échec, afin d'obtenir une classe de modèles de Thomas respectant la dynamique du modèle initial.
De même, la première étape de l'inférence peut, en outre des arcs positifs et négatifs du GI, produire des arcs non-signés lorsque certaines régulations ne peuvent être résumés à de simples activations ou inhibitions.
Cependant, dans la première version de ce travail, cette première étape ne pouvait être rattrapée,
car un tel arc non-signé ne pouvait être exploité par l'étape suivante, faisant alors échouer l'inférence à mi-parcours.

L'un des objectifs de l'enrichissement de ce travail a été de pouvoir exploiter les inférences d'arcs non-signés.
Pour cela, nous proposons une sémantique étendue du modèle de Thomas, comprenant des arcs positifs ($+$), négatifs ($-$) et non-signés ($\uns$).
Cette extension est possible du fait que le signe des arcs n'impacte pas la dynamique du modèle ; seule la paramétrisation spécifie la direction d'évolution du modèle dans chaque état.
Cependant, de la même façon qu'un arc positif (resp.~négatif) modélise sans ambiguïté une activation (resp.~une inhibition),
un arc non-signé modélise une régulation donc la nature est indéterminée ou plus complexe qu'une simple activation ou inhibition.
Étant donné ce nouvel outil, l'inférence d'un GI est possible dans tous les cas de figure, et un seuil est inféré quel que soit le signe.
L'inférence des paramètres est alors possible malgré la présence d'arcs non-signés étant donné que ceux-ci n'influent pas sur la dynamique.
Enfin, l'énumération des paramétrisations compatibles a aussi été revue afin d'être adaptée à cette nouvelle sémantique.
La compatibilité est assurée par des contraintes supplémentaires qui assurent notamment qu'un paramètre est utile (hypothèse d'activité) et cohérent avec le type de la régulation (hypothèse de monotonicité), en plus de respecter la dynamique du Process Hitting.
Le signe des arcs entre ainsi en compte dans cette énumération ;
cependant, les arcs non-signés n'apportent pas de contrainte à ce niveau afin de permettre toutes les réponses possibles, étant donné que leur nature exacte est indéterminée.
L'implémentation précédemment réalisée dans le cadre de ce travail a été adaptée, rendant l'inférence conclusive dans un plus grand nombre de cas.
Les résultats sont toujours produits en moins d'une seconde sur un ordinateur de bureau pour des modèles de 20 et 40 gènes.

Cette révision est aussi l'occasion d'enrichir la rédaction de discussions et d'explications supplémentaires à propos de certains points n'ayant pas été abordés dans la première version, faut de place.

\chapter{Perspectives}

\todo{Revoir !!!!
\begin{itemize}
  \item Priorités multiples avec précédence
  \item Traduction CIF2PH (Carito)
  \item Multiplexes avec délais (Morgan)
\end{itemize}
}



\section{Données chronométriques}
À partir de modèles algébriques tels que le Process Hitting ou le modèle de Thomas, il est possible d'extraire des propriétés à caractère chronologique,
par l'étude du Graphe des États ou à l'aide d'analyses statiques permettant de conclure quant à des atteignabilités successives de processus.
Cependant, il peut être nécessaire de contraindre ou d'étudier un système en fonction de données chronométriques, portant sur la durée des processus mis en jeu plutôt que sur leur simple succession.
L'une des perspectives de ce sujet serait ainsi d'étudier les possibilités d'étendre la sémantique du Process Hitting afin de pouvoir y inclure des données chronométriques discrètes ou continues.

%Une telle extension pourrait revêtir différents buts.
L'un des principaux buts d'une telle extension est l'inférence des délais au sein d'un modèle afin d'obtenir des résultats temporels sur certains phénomènes difficilement reproductibles ou observables expérimentalement.
L'étude du cycle circadien, qui fait partie des thématiques étudiées par MeForBio \textit{via} notamment le projet CirClock, est un exemple de domaine qui bénéficierait directement d'un tel développement.
D'autres objectifs peuvent être dégagés de cette extension, comme l'ajout d'un système d'horloges à la modélisation permettant la représentation de comportements plus complexes et dépendant du temps.

Cette perspective s'inscrit dans la collaboration en cours de développement avec l'équipe du professeur Katsumi Inoue du National Institue of Informatics,
dont le but est de développer des outils permettant l'inférence de données chronométriques à l'aide d'outils SAT.
%dans le cadre d'un projet STIC-Asie SATTIBIS

%Projet avec le Japon
%+ sujet doctorat



\section{Réflexion sur les propriétés d'atteignabilité}

L'une des forces principales du Process Hitting repose en les outils d'analyse statique permettant de vérifier des propriétés d'atteignabilité.
Cependant, ces outils s'appliquent uniquement à la sémantique de base du Process Hitting ;
les ajouts à la sémantique entraînent généralement l'impossibilité d'utiliser ces outils.
Une perspective intéressante serait ainsi de mener une réflexion approfondie sur l'application de ces outils à d'autres sémantiques.

Le Process Hitting à priorités fixes, au moins dans certaines restrictions de son utilisation, pourrait notamment bénéficier de ces outils.
Leur application aux autres sémantiques présentées à la section \todo{xx} pourrait pourrait être intéressante sous réserve que ces sémantiques présentent un intérêt pour certaines modélisation.
Enfin, cette réflexion devra être menée durant tout le processus de développement d'une sémantique de Process Hitting intégrant des données chronométriques,
afin d'en permettre une exploitation efficace.

\chapter{Activités scientifiques}

\section{Publications}\label{sec:publications}

\subsection{Publications 2012/2013}
\todo{CS2Bio}

\todo{JDOC ?}

\subsection{Publications précédentes}\label{ssec:publications}
\todo{Alléger}

\paragraph{Article accepté en conférence}
\begin{itemize}
\item[] \textbf{Maxime Folschette}, Loïc Paulevé, Katsumi Inoue, Morgan Magnin, Olivier Roux.
Concretizing the Process Hitting into Biological Regulatory Networks,
in : CMSB'12\!\!: Proceedings of the 10th International Conference on Computational Methods in Systems Biology,
London, UK, ACM, octobre 2012.
\end{itemize}
Une version préliminaire de cette publication est disponible en annexe de ce document.

\paragraph{Article soumis en workshop}
\begin{itemize}
\item[] \textbf{Maxime Folschette}, Loïc Paulevé, Katsumi Inoue, Morgan Magnin, Olivier Roux.
Abducting Biological Regulatory Networks from Process Hitting models,
in : LDSSB'12\!\!: ECML/PKDD 2012 Workshop on Learning and Discovery in Symbolic Systems Biology,
University of Bristol, UK, septembre 2012
\end{itemize}

\paragraph{Précédente publication en journal} dans le domaine de la fusion nucléaire
\begin{itemize}
\item[] Andrea Murari, Didier Mazon, Michela Gelfusa, \textbf{Maxime Folschette}, Thibaut Quilichini et collaborateurs EFDA-JET.
Residual analysis of the equilibrium reconstruction quality on JET, \textit{Nuclear Fusion},
volume 51, numéro 5, avril 2011, DOI 10.1088/0029-5515/51/5/053012.
\end{itemize}



\section{Exposés invités}

\section{Exposés avec proceedings}
\todo{MOVES}

\todo{ASSB}

\section{Exposés précédents}
\todo{Alléger}

\paragraph{Réunion du groupe de travail ANR BioTempo/G2 : « Des dynamiques discrètes à des dynamiques continues (modèles hybrides) »}
juin 2012, Nantes, France
\begin{itemize}
\item[] Concretizing Process Hitting models into Biological Regulatory Networks with Thomas' formalism using ASP
\end{itemize}

\paragraph{Séminaire informel AED}
juin 2012, Nantes, France
\begin{itemize}
\item[] Modeling and Analysis of Large Biological Regulatory Networks thanks to the “Process Hitting” Framework
\end{itemize}

\paragraph{Fourth CSPSAT \& ASP Seminar}
mai 2012, Kobe, Japon
\begin{itemize}
\item[] Concretizing Process Hitting models into Biological Regulatory Networks with Thomas' formalism using ASP
\end{itemize}

\paragraph{KUBIC-NII Joint Seminar on Bioinformatics 2012}
avril 2012, Kyoto, Japon
\begin{itemize}
\item[] Translating Process Hitting models to Thomas' modeling with ASP
\end{itemize}

\paragraph{The 8th Meeting on Inference-based Hypothesis-finding and its Application to Systems Biology}
mars 2012, Kanazawa, Japon
\begin{itemize}
\item[] Modeling and Analysis of Large Biological Regulatory Networks thanks to the Process Hitting Framework
\end{itemize}



\section{Collaboration}

\paragraph{Katsumi Inoue} National Institute of Informatics, Tokyo, Japon

Séjour AtlanSTIC de trois mois (mars à mai 2012) financé par la fondation Centrale Initiatives et le National Institute of Informatics,
portant sur l'inférence du modèle de Thomas sous-jacent à un Process Hitting.
Développement de l'outil \texttt{ph2thomas} permettant cette inférence à l'aide d'Answer Set Programming.

\chapter{Autres activités}

\section{Enseignement}

Activité complémentaire d'enseignement (monitorat) à l'École Centrale de Nantes ; les heures de TP sont décomptées comme des heures équivalent TD.

\todo{Actualiser}

\bigskip

\noindent
\textbf{Méthodes logicielles (MELOG) :} 2\textsuperscript{e} année (semestre 7)\\
Programmation orientée objet, structures de données et langage Java\\
\textbf{Responsable :} Guillaume MOREAU
\begin{itemize}
  \item 2 groupes de TP, soit 28 heures
\end{itemize}

\bigskip\noindent
\textbf{Algorithmique et programmation (ALGPR) :} 1\textsuperscript{ère} année (semestre 6)\\
Introduction à l'algorithmique et applications au langage C\\
\textbf{Responsable :} Vincent TOURRE
\begin{itemize}
  \item 2 modules de TD, soit 4 heures
  \item 2 modules de TP, soit 4 heures
\end{itemize}

\bigskip\noindent
\textbf{Projet d'application (PAPPL) :} 3\textsuperscript{e} année, option informatique (semestre 8)\\
Projet d'application informatique\\
\textbf{Responsable :} Didier LIME
\begin{itemize}
  \item encadrement de 2 projets pour un total de 3 étudiants en co-encadrement, soit 4,6 heures
\end{itemize}

\bigskip\noindent
\textbf{Projet d'application (\todo{PGROU}) :} 3\textsuperscript{e} année, option informatique (semestre 8)\\
Projet d'application informatique\\
\textbf{Responsable :} Didier LIME
\begin{itemize}
  \item encadrement de 2 projets pour un total de 3 étudiants en co-encadrement, soit 4,6 heures
\end{itemize}

\bigskip\noindent
\textbf{Projet d'application (\todo{R\&D}) :} 3\textsuperscript{e} année, option informatique (semestre 8)\\
Projet d'application informatique\\
\textbf{Responsable :} Didier LIME
\begin{itemize}
  \item encadrement de 2 projets pour un total de 3 étudiants en co-encadrement, soit 4,6 heures
\end{itemize}



\section{Modules d'apprentissage}
\todo{MOVES et ASSB ?}

\todo{Présenter sa thèse en temps limité}

\todo{Initiation au journalisme scientifique}

\todo{Anglais pour la recherche}




\section{Responsabilité administrative}
Membre actif de l'\emph{Association des Étudiants en Doctorat} sur le campus de l'École Centrale de Nantes (AED).


\bibliographystyle{plain}%alpha}
\bibliography{biblio}

%\newpage
%\todo{ANNEXES : ARCS NEUTRALISANTS + PH2THOMAS}

\end{document}
